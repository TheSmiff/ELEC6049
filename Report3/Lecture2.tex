% !TeX spellcheck = en_GB
% !TeX root = Report.tex
\phantomsection
\addcontentsline{toc}{subsection}{Lecture 2 - Mr Tim Whitcher, London Power Associates}
\subsect{Lecture 2 - Mr Tim Whitcher, London Power Associates}

%Tim is currently working on a project on the London Underground
Mr Tim Whitcher is a contractor, providing management support.
Mr Whitcher's current project involves the upgrade of the Northen and Jubilee Lines on the London Underground. 

Transport for London (TfL) is a govenment body who are responsible for public transport in London.
Their responsibilities include the London Underground, bus service and the Boris bikes, amongst other areas \cite{tfl:about}.
Around 24 million journeys are made every day accross the TfL network.
In order to support this capacity of commuter, TfL must supply an efficient service, and their use of technology is key to this.

The Oyster Card was introduced in 2003 and now around 80\% of all journeys are paid for using the contactless payment card \cite{oyster:wiki}. 
The card supports a standard, initially written by NXP. 
Although there is controversy over the security of the Mifare RFID cards \cite{molnar2004privacy}, the Oyster card does not use a second generation card which is significantly more secure \cite{rfid:security}. 
By exploiting this standard, and the ease of use of the technology, large queues are avoided on public transport, getting commuters to their destination much quicker. 
Although the cost of the travel is less to the commuter, the service the Oyster card enables TfL to provide is a clear benefit.

There is also expansion to look into the use of technology on buses to help prevent accidents \cite{tfl:bus}. 
Although this is not fully live yet, it looks promising to prevent collisions with bicycles or pedestrians that the bus driver cannot see.
If less accidents are caused, the reputation of the bus system will be increased, and a quicker service will be supllied to the commuter.
The idea of electric buses is also a potential technology for London's shorter bus routes \cite{guardian:bus}.

Overall, the technology used by TfL does not directly create income for the transport system.
However, by exploiting technology, the whole London Transport system can be made seemless, attracting more commuters to use it.
