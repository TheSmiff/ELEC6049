% !TeX spellcheck = en_GB
% !TeX root = Report.tex
\phantomsection
\addcontentsline{toc}{subsection}{Case 1 - Transport for London: Crossrail Ltd}
\subsect{Case 1 - Businesses connected to Crossrail Ltd}

Crossrail Ltd (CRL) is a subsidiary of TfL.
Originally a joint venture with the Department for Transport (DfT) their goal is to develop transport links throughout the South East and secure London's financial excellence in Europe~\cite{crossrail:about}.
In February $2005$ CRL proposed the \emph{Crossrail Bill} which became the \emph{Crossrail Act} in July $2008$.
This supports construction of a ``railway transport system running from Maidenhead\dots and Heathrow Airport\dots through central London to Shenfield''~\cite{crossrail:act}.
As the largest civil engineering project in Europe CRL has awarded direct contracts worth upto \pounds $5.5$bn benefiting over $17,000$ businesses~\cite{crossrail:suppliers}.
In this group of small enterprises and large companies which have benefited most and did they adapt their business model specifically to allign with governance?  

In February $2014$ a Canadian company Bombardier Inc. won a \pounds $1$bn contract, over Hitachi, to deliver and maintain $65$ new trains and a depot for the line~\cite{bbc:bombardier,tfl:bombardier}.  
Bombardier has been providing mass transit solutions since $1974$ and growing steadily since so this opportunity is part of their core competence~\cite{bombardier:about}.
The company owns a passenger rolling stock plant in Derby and in $2011$ cut over a third of staff there as they failed to win a contract for the Thameslink service over Siemens \citeneeded{}.
At this point the entire future of the plant and subsequently a remaining $1,500$ jobs in Derby lay on securing the contract for the Crossrail project.

PODFather is a company that manufacture rugged handheld devices for use in logistics, earthworks and field services.
In March $2013$ the company revealed CRL is using ``PODFather for Field Force capture of Vehicle Bookings data''~\cite{podfather:crl}.
CRL is just one of their many customers which includes large enterprises such as Alstom. 


CRL estimates they have provided a \pounds $42$bn benefit to the construction industry in the UK.
This benefits businesses in the UK because the governments objective is to recycle money in in the local economy.
It is therefore a great time to be a earthworks/train company based in UK; even more so with High Speed 2 (HS2) project on the horizon.

