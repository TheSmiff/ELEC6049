% !TeX spellcheck = en_GB
% !TeX root = Report.tex
\phantomsection
\addcontentsline{toc}{subsection}{Case 2 - Opportunities Presented by Government Policy on Nuclear New Build}
\subsect{Case 2 - Opportunities Presented by Government Policy on Nuclear New Build}

The UK has a rich nuclear heritage and has been generating commercial quantities of power from nuclear power plants since Calder Hall opened in 1957 \cite{NDA2007}. 
The nuclear industry is one of the most heavily regulated industries in which to operate and opportunities to profit from technology are strongly coupled to the position of the government and their subsequent policy decisions. 
Companies that can react quickly to maximise available opportunities will reap the largest rewards from new nuclear.

Government policy has been largely anti-new-build since the last nuclear plant was commissioned at Sizewell B in 1992. 
This plant was envisioned to be the first of a fleet of new pressurised water reactors, but only Sizewell B was ever built \cite{WNA2014}. 
The new build supply chain has since diminished, with construction expertise and nuclear specific skills depleting.
However, opportunities to profit from nuclear technology did not diminish entirely in this period. 
The UK's ageing fleet of nuclear reactors contributed 19\% of UK electricity supply in 2012 \cite{WNA2014}.  
The French utility EDF made \pounds880m profit in 2012 from their UK nuclear operation including eight plants totalling 8.7GW capacity \cite{EDF2012}. 
Evidently there were sparse but significant opportunities in the industry prior to 2008.

Government nuclear policy has gradually evolved since privatisation of the energy industry in 1995. 
Labour’s 1997 manifesto was strongly against new nuclear \cite{Birmingham2012}. 
The 2003 Energy White Paper set targets for the reduction of carbon emissions by 60\% by 2050. 
While new nuclear build was not ruled out, it was not supported for economic reasons \cite{WP2003}. 
Against a backdrop of rapidly increasing fossil fuel prices and increasing concerns over energy security, the 2007 Energy White Paper started the public consultation on new nuclear build \cite{WP2007} resulting in the plans to promote new nuclear build in the 2008 White Paper on nuclear \cite{WP2008}.
This turnaround in Government policy presents a new opportunity for private companies to profit from investments in new nuclear. 
The new policy is expected to attract \pounds40bn investment up to 2025 \cite{Birmingham2012}.

Currently there are five UK sites under active development by three companies each intending to deploy a different reactor technology:
\begin{table}[!htb]
\caption{UK Nuclear New Build Summary}
\label{table:NNB}
\begin{center}
\begin{tabular}{p{0.3\textwidth}p{0.15\textwidth}p{0.15\textwidth}p{0.28\textwidth}}
\toprule
\textbf{Company} 		&	\textbf{Parent Company}	& 	\textbf{Sites} 	& 	\textbf{Technology} \\ \toprule
NNB GenCo (Nuclear New Build Generation Company)&EDF Energy & Hinkley Point \& Sizewell&Areva EPR (European Pressurized Reactor)\\
Horizon Nuclear Power&Hitachi& Wylfa \& Oldbury& Hitachi-GE ABWR (Advanced Boiling Water Reactor)\\
NuGeneration&Toshiba \& GDF Suez&Moorside (Sellafield)&Westinghouse AP1000\\
\bottomrule
\end{tabular}
\end{center}
\end{table}

 

Nuclear supply chain renewal
