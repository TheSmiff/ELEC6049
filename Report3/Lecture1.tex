% !TeX spellcheck = en_GB
% !TeX root = Report.tex
\phantomsection
\addcontentsline{toc}{subsection}{Lecture 1 - Fabrice Perrot, Alstom}
\subsect{Lecture 1 - Fabrice Perrot, Managing Director Alstom Grid Research and Technology Centre}
Alstom Grid is one of four technology sectors making up the Alstom group.
The grid sector was formed in 2010 when Alstom acquired Areva's transmission and distribution arm \cite{AlstomGridInfo} for \euro2.3bn \cite{AlstomAreva}. 
With a worldwide presence and over 20,000 employees \cite{AlstomGridInfo}, Alstom Grid has a significant presence in the High Voltage grid sector as outlined in Figure \ref{figure:AlstomGrid}
Projects undertaken can be very large scale, including a \euro62m involvement in floating substations in the German north sea \cite{AlstomGridInfo}, building the Gulf's largest High Voltage Direct Current (HVDC) converter station \cite{AlstomGulf}, and delivering the converter transformers for the world's longest HVDC transmission scheme at Rio Madeira in Brazil \cite{AlstomRio}.

\begin{figure}[!h]
\centering
\includegraphics[width = 0.7\textwidth]{Figures/AlstomGrid.jpg}
\caption{Alstom Grid Services - reprinted from \cite{AlstomGridInfo}}
\label{figure:AlstomGrid}
\end{figure}

The electrical transmission infrastructure is an often overlooked service critical to maintaining a western lifestyle.
The supergrid infrastructure in the UK was built largely in the 1950s to cater for the connection of the large coal and gas power stations to the large load centres \cite{NationalGrid75}.
The nature of the electricity industry is fundamentally changing.
An increase in embedded generation and the remoteness of renewable sources leaves the grid network in need of significant investment.

The drive towards renewables is driven by government policy.
The Climate Change Act 2008 sets out targets to reduce carbon emissions by 80\% of 1990 levels by 2050 \cite{ClimateChangeAct08}.
In order to meet this commitment in any meaningful sense, the electricity sector must be decarbonised considerably earlier than 2050, so that space heating and transport sectors can transition to a carbon-neutral energy vector.
There is a unified approach in many developed nations, particularly in Europe, promoting renewable energy development.
It is this legislation that is driving the change in the design basis for almost all modern transmission and distribution infrastructure.

The opportunities to profit from developing and installing new grid infrastructure is huge.
In the UK, the cost of network reinforcements is estimated at \pounds8.8bn \cite{NetworkStrategyGroup}. 
A large majority of the investment will be required for the connection of offshore wind, usually via HVDC links \cite{NetworkStrategyGroup}.
Similar opportunities to profit from government infrastructure investment exist worldwide.
The DeserTec solar project aims to connect the North African grid to the European Super Grid providing access to cheap, carbon-free concentrating solar power plants. 
The scale of the project can be appreciated in Figure \ref{figure:Desertec}.
The project is an extremely challenging endeavour considering the political cooperation involved, but should the project proceed, the investment could total \$560bn \cite{DesertTec}.
Alstom grid should be in a position to provide HVDC equipment for this project, which could present significant profits.

\begin{figure}[!h]
\centering
\includegraphics[width = 0.7\textwidth]{Figures/Desertec.jpg}
\caption{Potential structure of the DeserTec HVDC network extending over the Middle East, North Africa and Europe - reprinted from \cite{DesertTec}}
\label{figure:Desertec}
\end{figure}

Considering these opportunities, Alstom's acquisition of Areva Transmission and Distribution seems a clever strategic investment.
The acquisition of a core competence in HVDC at a time when global government policy is promoting renewed investment in the sector is a clever move to align with government policy in order to profit.





