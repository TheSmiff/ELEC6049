% !TeX spellcheck = en_GB
% !TeX root = Report.tex
\phantomsection
\addcontentsline{toc}{section}{Conclusions}
\sect{Conclusions}

The funding behind governance driven technological ventures is incredibly large.
Case studies considered deal with sums of money in the order of billions of pounds.
Involvement with projects rooted in government can obviously generate business but does this mean more profit?
Profit is a fraction of turnover but bidding for contracts can drive down margins which are traded off for potential repeat business or publicity for high profile projects.  

CRL is directly driven by government unlike the remaining two case studies.
New build provides permission but not drive for nuclear development in the energy sector which is achieved by private companies.
4G mobile broadband can be consider a resource belonging to the state where governance permits auction.
These sectors could be, albeit naively, ranked of priority by involvement of the government.
In order of transport, energy then communications but is this the case?

Governance is required for managing a national electromagnetic spectrum and auctions are schedule by government, or the regulatory committee acting on behalf of government, but leasing requires large amounts of funding.



