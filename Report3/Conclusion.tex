% !TeX spellcheck = en_GB
% !TeX root = Report.tex
\phantomsection
\addcontentsline{toc}{section}{Conclusions}
\sect{Conclusions}

The funding behind governance driven technological ventures is incredibly large.
Case studies considered deal with sums of money in the order of billions of pounds.
Involvement with projects rooted in government can obviously generate business but does this mean more profit?
Profit is a fraction of turnover but bidding for contracts can drive down margins which are traded off for potential repeat business or publicity for high profile projects.  

CRL is directly driven by government unlike the remaining two case studies.
New build policy gives permission but not explicit drive for nuclear development in the energy sector which is achieved by private companies.
4G mobile broadband can be considered a resource belonging to the state where governance permits auction to make a profit themselves.
These cases could, albeit naively, be ranked of priority by involvement of the government.
In order of transport, energy then communications but is this the case?
Energy is critical infrastructure but with budgets so high perhaps any funding the government could provide wouldn't make a dent in what is required?

Governance is required for managing a national electromagnetic spectrum and auctions are scheduled by government, or the regulatory body acting on behalf of government, but leasing requires large amounts of funding.
This required for the business model and all competitors are hindered by the same issue but gaining early access is a potential route to increased profit.
Companies connected to CRL have only benefited whether through chance or by design and new build has revived a declining subset of the energy industry.
Aligning with governance can therefore been seen as a positive influence on financial performance.
