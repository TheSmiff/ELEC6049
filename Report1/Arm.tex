\phantomsection
\addcontentsline{toc}{subsection}{ARM}
\subsect{ARM - John Biggs - Consultant Engineer and Co-Founder}
%Overview of each of the talks and an assessment of the information presented in light of the report topic 'How does technology make profit?' 
ARM was founded in 1997 as a joint venture between Apple and Acorn. %1990????
The primary function of ARM is to design processor architectures.
Robin Saxby, ARM's first CEO, developed the innovative business model that is partially responsible for ARM's success.
ARM licenses their architecture IP to chip manufacturers, as opposed to the manufacturing and selling a physical product themselves. 
As well as a license, the partnership model includes a royalty system. 
This gives ARM a constant revenue stream, allowing the funding of further research.
They are now the world's leading semiconductor IP company and have shipped 50 billion devices to date.

ARM were able to grasp the low power market ahead of larger CPU firms such as Intel. 
1988 saw handheld PDAs starting to emerge, prompting several low power CPU designs from different manufacturers.
The ARM610 chip featured in the Apple Newton Message Pad, prompting the joing venture spin out.
ARM has significant expertise in low-power design and this core competency has given ARM a competitive advantage as the transition to the mobile age and the Internet of Things continues.

Building on this core expertise in low power architecture design, ARM have developed a focused but varied range of products.
In 1997, the company expanded into memory, video and I/O controllers to support the processors they designed.
The latest processors can be found in a range of devices, from $mm^{3}$ sized devices, to $km^{3}$ sized. 
ARM is also currently expanding into the microserver market.
By applying the low-power design considerations in a server environment, a significant reduction in server running costs can be achieved.
Although a relatively young company, some degree of related diversification has already occured at ARM.
When coupled with technical excellence, related diversification will allow ARM to continue to exploit tehcnology for profit.


