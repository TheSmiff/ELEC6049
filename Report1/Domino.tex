\subsect{Domino Printing Sciences PLC - Carl Reynaud - Director of Hardware Development}
%Overview of each of the talks and an assessment of the information presented in light of the report topic 'How does technology make profit?' 
Domino Printing Sciences PLC designs, manufactures and markets a range of printing equipment for a wide range of applications \cite{DominoAnnual}. 
In 2013 Domino achieved a global turnover of \textsterling335.7m from operations in over 120 countries \cite{DominoFactsheet}. 
Domino was founded in 1978 \cite{DominoFactsheet} as a spin out following a project at technology consultancy Cambridge Consultants.
Graeme Minto led the project developing continuous inkjet (CIJ) technology, and consequently founded Domino Printing Sciences following the departure of the project's client \cite{goffin2010innovation}.
This is an example of a company starting with a core technological innovation as the focus, as asserted by Johnson in \cite{ johnson2008exploring}.

By the mid-90s CIJ was reaching technological limits of exploitation. 
Despite the many advantages of the technology, customers wanted higher resolution images in bigger sizes than were possible with CIJ. Hence Domino began an extensive review of alternative technologies including laser printing, `binary' inkjet and `drop-on-demand' technology.
During the 90s, Domino built on the expertise built from CIJ technology to expand and diversify to become a multi-technology company. 
This diversification into other related technologies have enabled Domino to retain their existing customer base while entering new and emerging markets \cite{goffin2010innovation}.
Instead of just printing date codes, Domino has diversified into technologies to print patterns for laminates, ceramic floor tiles, complex food labels, cable labelling and more.
This is an example of a company adopting related diversification as a corporate strategy, enabling continued and extensive profit from technology.


