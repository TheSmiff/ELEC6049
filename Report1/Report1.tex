%% ----------------------------------------------------------------
%% Specification.tex
%% ---------------------------------------------------------------- 
\documentclass{elec6049Report}     % Use the Article Style
%\usepackage{cite}            % Use Natbib style for the refs.
\removecolourlinks    % Uncomment this command to remove colour from any links
%% ----------------------------------------------------------------
%\usepackage[disable]{todonotes}
\usepackage{todonotes}
\usepackage{multirow}
\newcommand{\inote}[1] {\todo[inline]{#1}}
\usepackage{lipsum}

\begin{document}

\frontmatter
\reportnumber{1}
\title      {How Does Technology Make Money? - Established Companies}
\authors    {
\texorpdfstring{\href{mailto:hl13g10@ecs.soton.ac.uk}{Henry S. Lovett}}{Henry S. Lovett},
\texorpdfstring{\href{mailto:ajr2g10@ecs.soton.ac.uk}{Ashley J. Robinson}}{Ashley J. Robinson},
\texorpdfstring{\href{mailto:tjs1g10@ecs.soton.ac.uk}{Thomas J. Smith}}{Thomas J. Smith}}

\groupnumber{14}

\maketitle

\begin{abstract}
An abstract of not more than 200 words is required. It should contain not only a description of the subject and scope of coverage, but also identify the focus of the report.  
\end{abstract}

\mainmatter
\sect{Introduction}
%This should cover the general concepts and principles relevant to the report \cite{kekale2007successful}, explain constructive \cite{ansoff1957strategies}, \cite{johnson2008exploring} and destructive technologies, compliance or similar in the context of electrical/ electronic/electromechanical technologies [1]). The purpose of the introduction is to 'set the scene'.  

In the context of this report, the definition of an established company is inspired by Kek{\"a}le et al \cite{kekale2007successful}. 
Established companies are those that have been successful for a number of years, with a defined product range for a specific consumer market. 
As a company develops and expands, a clear corporate strategy is required to enable continued profitability and further growth.
In this report, it is proposed that related diversification is the optimal corporate strategy for an established business, ensuring the company continues to profit from technology.

Diversification is a corporate strategy set on the simultaneous departure from the present product line and market structure \cite{ansoff1957strategies}.
Businesses have differing approaches to diversity broadly put into three categories; specialism, related diversification and unrelated diversification. 
%A specialist company is one that is yet to diversify from it's core product and market \cite{johnson2008exploring}. 
Many companies start with a breakthrough product or technological innovation, and a specialised strategy chooses to keep this product as the sole focus of the business \cite{johnson2008exploring}.
At the other end of the spectrum is unrelated diversification.
This strategy involves a company choosing to depart completely from the sector it initially operated in \cite{johnson2008exploring}, and is often referred to as the conglomerate strategy.
In between these two extremes is related diversification, where a company enters a new but related market using the skills base and knowledge they have gained in their current field \cite{johnson2008exploring}. 
Although argued to be contextual and subjective, it is generally the agreement of literature that related diversification produces better company performance \cite{johnson2008exploring, SMJ:SMJ334, SMJ:SMJ82}. 
This claim is examined in this report through examples presented both in the invited talks and through individual research.



%Many technology based companies start with a key breakthrough technology at the core of the business.
% For example, Google started with it's prolific search engine, Sony started building simple radios and Amazon started with the online market place. \inote{References/Better examples.}




\sect{Section 1 - Invited Talks} \label{Sect1}
This section contains a reflection on each of the invited talks in the light of the hypothesis of this report. Full notes from the lectures have been recorded in Appendix 4.

\subsect{Domino Printing Sciences PLC - Carl Reynaud - Director of Hardware Development}
%Overview of each of the talks and an assessment of the information presented in light of the report topic 'How does technology make profit?' 
Domino Printing Sciences PLC designs, manufactures and markets a range of printing equipment for a wide range of applications \cite{DominoAnnual}. 
In 2013 Domino achieved a global turnover of \textsterling335.7m from operations in over 120 countries \cite{DominoFactsheet}. 
Domino was founded in 1978 \cite{DominoFactsheet} as a spin out following a project at technology consultancy Cambridge Consultants.
Graeme Minto led the project developing continuous inkjet (CIJ) technology, and consequently founded Domino Printing Sciences following the departure of the project's client \cite{goffin2010innovation}.
This is an example of a company starting with a core technological innovation as the focus, as asserted by Johnson in \cite{ johnson2008exploring}.

By the mid-90s CIJ was reaching technological limits of exploitation. 
Despite the many advantages of the technology, customers wanted higher resolution images in bigger sizes than were possible with CIJ. Hence Domino began an extensive review of alternative technologies including laser printing, `binary' inkjet and `drop-on-demand' technology.
During the 90s, Domino built on the expertise built from CIJ technology to expand and diversify to become a multi-technology company. 
This diversification into other related technologies have enabled Domino to retain their existing customer base while entering new and emerging markets \cite{goffin2010innovation}.
Instead of just printing date codes, Domino has diversified into technologies to print patterns for laminates, ceramic floor tiles, complex food labels, cable labelling and more.
This is an example of a company adopting related diversification as a corporate strategy, enabling continued and extensive profit from technology.

\subsect{Arm - John Biggs - Consultant Engineer and Co-Founder}
Overview of each of the talks and an assessment of the information presented in light of the report topic 'How does technology make profit?' 

\subsect{Imagination Technologies - Dr David Knox - Senior Director Software Engineering}
Overview of each of the talks and an assessment of the information presented in light of the report topic 'How does technology make profit?' 

\subsect{Reflection on Common Themes}
Short conclusion identifying the common themes, or conflicting strategies from the talks.
Summarise in the light of our hypothesis.


\sect{Section 2 - Diversification for Profitability}
Section two of the report should contain researched examples of electronics/electrical companies and their technologies which illustrate your group’s point of view regarding the report topic.   (approximately 2000 words).
General gist of so far.

Amazon as a case study with online retail, kindle, automated delivery etc etc.

Google as a case study, similarities with Amazon with core as search engine and diversification along common theme of making data available everywhere. Although unclear as to current strategy with robotics and AI stuff, perhaps suggest preparing for the internet of things?

Intel as a negative case study in that they have been dominant in the Desktop microprocessor market for several years, but now profitability dropping due to change in market conditions and limited diversity. Discuss spreading risk and synergism.

Do a nice conclusion




\backmatter
\bibliographystyle{ieeetrans}
\bibliography{Report1Bib}

\appendix
\textbf{\uppercase{Appendix}} \par
\sect{1. Team Contributions}
\begin{center}
\begin{longtable}{|>{\raggedright\arraybackslash}m{0.2\textwidth} | m{0.75\textwidth} |} \hline
\textbf{Team Member} & \textbf{Contribution} \\ \hline
\endhead
\texorpdfstring{\href{mailto:tjs1g10@ecs.soton.ac.uk}{Thomas J. Smith}}{Thomas J. Smith} 23914254 & stuff \\ \hline
\texorpdfstring{\href{mailto:hl13g10@ecs.soton.ac.uk}{Henry S. Lovett}}{Henry S. Lovett} 23900091 & stuff \\ \hline
\texorpdfstring{\href{mailto:ajr2g10@ecs.soton.ac.uk}{Ashley J. Robinson}}{Ashley J. Robinson} 9999999 & stuff \\ \hline
\end{longtable}
\end{center}
\inote{AJR and HSL fill in your student number}

%\clearpage

\sect{2. Meeting Minutes - Kick-off Meeting}
\begin{center}
\begin{longtable}{| m{0.2\textwidth} | m{0.6\textwidth} |} \hline
\textbf{Purpose} & ELEC6049 Team Kick-Off Meeting \\ \hline
\textbf{Date and Time} & Tuesday 4th February 2014 at 12:00 \\ \hline
\textbf{Venue} & 3rd floor Zepler Building, Highfield Campus \\ \hline
\textbf{Participants} & TJS (Tom Smith), HSL (Henry Lovett), AJR (Ashley Robinson)\\ \hline
\textbf{Apologies} &None \\ \hline
\multirow{4}{*}{\textbf{Agenda}} & Assign Chair for this report. \\
 & Generate initial ideas for research. \\ 
 & Agree expectations of work and schedule. \\
 & Agree date and agenda of next meeting. \\ \hline
\end{longtable}
\end{center}

\subsect{Minutes of the Meeting}
\begin{center}
\begin{longtable}{| p{0.05\textwidth} |>{\raggedright\arraybackslash}p{0.15\textwidth} | p{0.5\textwidth} |>{\raggedright\arraybackslash}p{0.175\textwidth}|} \hline
\textbf{ID} & \textbf{Subject} & \textbf{Notes and Discussion} & \textbf{Action} \\ \hline
\endhead
1.0	&	Chair	&	The group decided that TJS should be the chair for this report. Both HSL and AJR are currently pursuing job applications, hence the logical choice that TJS be chair for this report. 	&   -	 \\ \hline
2.0	&	Case Study Ideas	&	TJS presented some ideas previously undertaken in a management module regarding Bowman's Strategy Clock, and the different strategies employed in the tablet market. The group discussed this as a case study and agreed it was interesting but obvious, but struggled to make a hypothesis. Discussions of reputation (Apple) and quality.	&-	 \\ \hline
3.0	&	Hypothesis 1	& HSL discussed Reputation - Companies such as Intel, Apple, Microsoft rely on their reputation as an established brand to sell their products. AJR proposed quality along the same lines.  & - \\ \hline
3.1	&	Hypothesis 2	& The group discussed collaboration in the context of ARM being fabless which enabled it to be more agile than Intel in the microprocessor market. AJR highlighted that ARM and Xilinix collaborate in the microprocessor and FPGA markets. Hypothesis is that by collaborating, established companies make money by reducing and spreading risks across multiple entities. Apple is an example contradicting the hypothesis. &	- \\ \hline
3.2	& Hypothesis 3 & TJS proposed that smart diversification was how an established company makes money. The group discussed Apple as an example with the iPhone, iPad and Mac ranges. Also Google with the search engine at the core, and Chrome and Android as diversification. Amazon with the online store at the core, and the Kindle e-reader and tablet as smart diversification along the same theme. A quick look in a textbook shows that this is called ``Related Diversification''. Microsoft with OSs and XBox. The group agreed that this was a possible hypothesis. &- \\ \hline
3.3 & Hypothesis selection & The group narrowed it down to collaboration or diversification as a topic. TJS chose to head toward the diversification hypothesis due to the strength and number of case studies. & \textbf{ALL A1.0} \textbf{All A3.0} \\ \hline
4.0 & Milestones & The group agreed to meet again on the \textbf{11th Feb} and that by then, most research should be complete, report stubs should be started and a clear hypothesis should be sought. The next meeting will be to put what we have together into a plan and to allocate work to complete the report by the tutorial on \textbf{18th Feb}. & \textbf{TJS A2.0} \\ \hline


\end{longtable}
\end{center}

\subsect{Action List}
\begin{center}
\begin{longtable}{| p{0.05\textwidth} | >{\raggedright\arraybackslash}p{0.15\textwidth} |  p{0.5\textwidth} | >{\raggedright\arraybackslash}p{0.175\textwidth}|} \hline
\textbf{ID} & \textbf{Action} & \textbf{Comments} & \textbf{Status} \\ \hline
\endhead
A1.0	&	Research	&	All to start research. Use Git Issue to highlight useful research to the group. Make notes of all sources.	& Open 4th Feb \\ \hline
A2.0	&	Introduction	&	Define an established company. Introduce hypothesis.	&	Open 4th Feb \\ \hline
A3.0	&	Case Study	&	Each to identify a case study for proposal to the group by next week.	&	Open 4th Feb	\\ \hline	
\end{longtable}
\end{center}

\subsect{Next Meeting: 11th Feb 2014, 12:00, Level 3 Zepler, Highfield Campus.}

%\clearpage

\sect{3. Meeting Minutes - Main meeting}
\begin{center}
\begin{longtable}{| m{0.2\textwidth} | m{0.6\textwidth} |} \hline
\textbf{Purpose} & ELEC6049 Report Meeting \\ \hline
\textbf{Date and Time} & Tuesday 11th February at 12:00 \\ \hline
\textbf{Venue} & 3rd floor Zepler Building, Highfield Campus \\ \hline
\textbf{Participants} & TJS (Tom Smith), HSL (Henry Lovett), AJR (Ashley Robinson)\\ \hline
\textbf{Apologies} & None \\ \hline
\multirow{5}{*}{\textbf{Agenda}} & Update on research status. \\
 & Discuss case study examples. \\ 
 & Read any report stubs as a group. \\
 & Identify and allocate work to finish report. \\ 
 & Agree next meeting. \\ \hline
\end{longtable}
\end{center}

\subsect{Minutes of the Meeting}
\begin{center}
\begin{longtable}{| p{0.05\textwidth} |>{\raggedright\arraybackslash}p{0.15\textwidth} | p{0.5\textwidth} |>{\raggedright\arraybackslash}p{0.175\textwidth}|} \hline
\textbf{ID} & \textbf{Subject} & \textbf{Notes and Discussion} & \textbf{Action} \\ \hline
\endhead
1.0	&	Project Description	&	stuff	&	\textbf{HL A1.0} \\ \hline
2.0	&	Intel Document	&	stuff	&	\textbf{TS A5.0} \\ \hline
3.0	&	Project Management	&	stuff &	\textbf{TS A7.0} \\ \hline
4.0	&	Research Areas	&	stuff &	\textbf{All A9.0} \\ \hline


\end{longtable}
\end{center}

\subsect{Action List}
\begin{center}
\begin{longtable}{| p{0.05\textwidth} | >{\raggedright\arraybackslash}p{0.15\textwidth} |  p{0.5\textwidth} | >{\raggedright\arraybackslash}p{0.175\textwidth}|} \hline
\textbf{ID} & \textbf{Action} & \textbf{Comments} & \textbf{Status} \\ \hline
\endhead
A1.0	&	 Spec.	&	action 	& Open 30th Sept \\ \hline
A2.0	&	GitHub Repo.	&	action	&	Open 30th Sept \\ \hline
A3.0	&	GitHub.		&	action	&	Open 30th Sept	\\ \hline
A4.0	&	Read the Spec.	&	action	&	Open 30th Sept \\ \hline
A5.0	&	Upload Intel Doc.	&	action	&	Open 30th Sept \\ \hline
A6.0	&	Read Intel Doc.	&	action	&	Open 30th Sept \\ \hline
A7.0	&	Setup meeting with Captec.	&	action	&	Open 30th Sept \\ \hline
A8.0	&	Gantt chart.	&	action	&	Open 30th Sept	\\ \hline
A9.0	&	Research.&	action  &	Open 30th Sept	\\ \hline	
\end{longtable}
\end{center}

\subsect{Next Meeting: 2nd Oct 13, following project brief with Captec.}

\clearpage

 
\sect{4. Notes from Invited Presentations} \label{Notes}
These notes are raw and not altered in any way from when they were taken from the invited presentation. These notes have been distilled and focussed through the lens of our report title and hypothesis to the content shown in section 1.
\subsect{Domino Printing Sciences PLC - Carl Reynaud - Director of Hardware Development}
Success - ``bringing new products into market and making a profit.''
The key question for any business is how to make profit in a sustainable manner. You need to be able to make profit, then keep on making a profit.

Background to Domino Printing - mission statement - we will achieve market recognition as the first choice global provider of coding, marketing and variable printing solutions delivering convenience, security and peace of mind. Key markets for Domino include;
\begin{itemize}
\item Date coding packaging - Slippery surfaces requires special technology.
\item Industrial printing and coding - this is on fast moving, small and hot plastics and requires a similar technology to date coding.
\item Printing and Mailing - Completely different technology required to the other two market areas.
\end{itemize}

Domino started with a key breakthrough technology in inkjet printing. This has led to diversification into laser and multijet technologies. The head office is in Cambridge. Domino manufacture around 1000-10,000 products per year, making it a low-medium volume business. This presents a number of challenges and opportunities. The technology portfolio ranges from ideas to mature technologies.

``Most ideas take 20 years to become an overnight success'' - Paul Saffo.

Four key points:
\begin{itemize}
\item You want what?
\item Celebrating Diversity
\item No absolutes
\item Keep updating the toolbox
\end{itemize}

You want what highlights that people are often the forgotten ingredient. It is key to know your customer. The customer often lacks the ability to articulate the problem properly and also lacks the vision to see the solutions. Identify what the customer really values (not necessarily the same as what they say they value) and why that is the case, as that is what customers will pay for. As an engineer, just because it is possible doesn't mean we should do it. Does the world need a better mouse trap? There is a significant skill in being able to capture and distill customer values and then apply that to a practical concept. A practical example was given of a customer specification asking for a printer weighing less than 25kg. Asking why this was the case broke the specification down to a real world requirement, that the printer should be able to be installed by one person to minimise installation costs. This requirement requires far more consideration than just weight and concerns shape, handling, regulation (lone working) and other factors. This is called a 5Y process. Another useful model to follow is the V-model.

Celebrating Diversity - don't try to put in what nature left out. changes in culture, education adn psychology. We need to identify how to work best with each other as a team. This involves knowing how you personally work, and identifying how others you work with behave so that you make best use of the diverse skillset in your team. Meyers Briggs type indicators are often used here. Diversity is particularly important when considering foreign markets. 

No absolutes - there will always be variations, but what is the failure zone. Process capability - understand variance. Need to identify suitable margins of error. Test products to make them fail - don't let the customer have a failed product as that can be disasterous. 

Keep updating the toolbox - Pareto and the 80:20 rule was mentioned, identify the 20\% of the work that delivers 80\% of the value etc. Failure mode effect and analysis provides a framework for risk analysis. 

%\inote{Questions and Answers (Ashley)}

Question and answer session:

"What's the biggest barrier you have encountered as an engineer?" - 
Not understanding how to apply technology as such to make it valuable for the customer. 

"How do the features of an engineering product relate to its success?" - 
Good products tend to have less features. 
It may be the case that less time can be spent fixing a parameter by researching customer preferences than implementing parameter adjusting functionality.
For example take the iPad versus a laptop.
Older generations will choose an iPad because there are less features therefore making it simpler to operate.

"Does Domino have any plans for expansion?" - 
We are currently invested in date code printing but are making a move to variable surface printing and also the printing of different materials.

"You mentioned how knowing your customer was important. How could a business suceed with emrging technologies when no record of the customer is already known?" -
There may exist parallel markets which can give some confidence of an expected customer base.
Focus groups and trade shows are a good way to test new technologies without fully deploying any products.
Constantly requesting customer input is also a good way to update the possible requirements which would lead to the design of a sucessful product.

"How do you measure success?" -
I look at technical forums where customers seek solutions to problems.

"Does domino have any plans to takeover smaller businesses?" -
Not at the present.
We tend to look for businesses that are making a profit rather the businesses which are producing future possible technologies.

"Have you thought of diversifying by making components you would otherwise outsource?" - 
Some of our print heads are brought in because the piezo electronics is too complicated for us to invest resources.
The mounting mechanisms however are designed internally but built externally. 
This allows to retain the intellectual property but not get involved with the machining required for implementation.
We look internally for skills that already exist onto which we could diversify rather than jumping to a completely new business. 
\end{document}
