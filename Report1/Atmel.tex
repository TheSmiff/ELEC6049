%  Atmel.tex
%  Document created by seblovett on seblovett-Ubuntu
%  Date created: Sun 16 Feb 2014 13:54:36 GMT
%  <+Last Edited: Sun 16 Feb 2014 17:01:45 GMT by seblovett on seblovett-Ubuntu +>
\phantomsection
\addcontentsline{toc}{subsection}{Atmel}
\subsect{Atmel Technologies Ltd.}

Atmel Technologies Ltd are a leading company in the design and manufacture of a variety of hardware products. 
The company was founded in 1984 by George Perlegos in the USA, a former Intel employee. 
It started with only \$30,000 in capital \cite{atmel:capital} and has since made \$150.93 million profit in Q4 of 2013 alone \cite{atmel:profit}, and is traded in the NASDAQ \cite{atmel:nasdaq}.
Atmel satisfies the criteria for being an established company - it has been successful for many years, and has a defined product range.
This case study will give a brief overview of the company, including an investigation into the diversity of Atmel, and whether this diversity has helped the success of the company.



Atmel's original expertise was non-volatile memory devices, primarily EEPROM memory.
Atmel sold it's memory devices to companies such as Nokia and Motorola.
However, in 1987, Intel started legal proceedings against Atmel.
Intel claimed the Atmel had infringed their patent, and Atmel did not fight the case \cite{atmel:intel}.
Instead of spending time and money on the law suit, Atmel redesigned their memory devices which subsequently outperformed and consumed less power than the Intel equivalents.
By being forced to innovate a new and better technology, Atmel gained a competitive advantage over the incumbent Intel, which was critical to the success of Atmel as an emerging company.
%As well as beating Intel in the EEPROM market, Atmel entered the flash memory market and out competed Intel there too.
%The patent case forced Atmel to redesign it's product, and can be seen as a, albeit small, diversification.
%This resulted in a much better product than Intel, giving Atmel the edge in the market.


%Since entering the memory market, Atmel also began to branch out to other markets. 
As Atmel became increasingly dominant in the memory market, new potential markets were explored.
Atmel licensed an architecture from ARM to enter microcontroller industry.
The company then combined their memory devices, both flash and EEPROM, with their AVR devices and the first microcontroller was released in 1996.
This is a logical extension of Atmel's core capabilities in memory, allowing Atmel access to a new and profitable marketplace.


%Not only did Atmel branch into relevant markets, but the company also acquired many other companies during their growth. 
%They also bought a number of foundries from other companies, and later sold the majority of them.
%Not only did the company gain different product areas, but they also sold parts of the company over time too. 
%This raises the argument that too much diversification could be bad.
%All acquisitions and sales are outlined below in chronological order.% and were found from \cite{atmel:acq1, atmel:acq2}
%\begin{enumerate}
%\item[1989] Purchased a Fab in Honeywell - enabled Atmel to conduct more R\&D %\cite{atmel:acq:}
%\item[1991] Acquired Concurrent Logic - an FPGA manufacturer. %\cite{atmel:acq:}
%\item[1994] Acquired Seeq Technologies - data communications and semiconductor devices%\cite{atmel:acq:}
%\item[1996] Acquired European Silicon Structures and Digital Research in Electronic Acoustics%\cite{atmel:acq:}
%\item[1998] purchased a part of TEMIC, a discrete semiconductor manufacturer. Included two fabs in France and Germany %\cite{atmel:acq:}
%\item[2000] Purchased a fab in North Tyneside from Siemens %\cite{atmel:acq:}
%\item[2005] Sold fab in Nantes, France \cite{atmel:acq:nantes05}
%\item[2006] Sold fab in Grenoble, France \cite{atmel:acq:grenoble06}
%\item[2007] Sold fab in North Tyneside, UK\cite{atmel:acq:tyneside07}
%\item[2008] Sold fab in Heilbronn, Germany\cite{atmel:acq:heilbronn08}
%\item[2008] Acquired Quantum Research Group to enter touch screen industry\cite{atmel:acq:qrg08}
%\item[2010] Sold fab and branch in Rousset, France\cite{atmel:acq:sms10}
%\item[2011] Sold D.R.E.A.M. %\cite{atmel:acq:}
%\item[2011] Acquired Advanced Digital Design S.A who specialised in communications with power lines \cite{atmel:acq:add11}
%\item[2012] Sold `DataFlash' products \cite{atmel:acq:sflash12}
%\item[2012] Acquired Ozmo Devices who made Wi-Fi products %\cite{atmel:acq:}
%\item[2013] Acquired Smart Metering products from IDT Corporation \cite{atmel:acq:smartmeter13}
%\end{enumerate}

Following the diversification into the microcontroller market, the company acquired a number of fabrication plants to support development in these areas.
The first acquisition was the Honeywell fabrication plant, increasing Atmel's R\&D capability \cite{atmel:acq1}.
Atmel then continued to diversify by aquiring further fabrication plants in a variety of locations.
These acquisitions were costly investments in large assets.
Although related to Atmel's product line, manufacturing was unrelated to Atmel's core competence of memory and microcontroller design.
These activities require vastly different expertise, skills and approach.
Atmel's experience and excellence in microcontroller and memory design had no bearing on the ability of the company to manage a fabrication plant.
As a result, this venture into manufacture ended as Atmel proceeded to sell the majority of it's fabrication plants \cite{atmel:acq:nantes05, atmel:acq:grenoble06, atmel:acq:tyneside07, atmel:acq:heilbronn08}.
This shows that not all diversification strategies are successful when unrelated to the core competence of the business.
It highlights the importance of understanding what makes a business successful in their current market, and the ability to identify related markets where these capabilities can be equally advantageous.

Atmel now design a number of hardware devices, testing them using their remaining R\&D foundry, and getting them manufactured for the market by external companies.
These devices range from microcontrollers for various applications, touch screens, Wi-Fi and security.
Keeping to their core competence of designing hardware has since enabled Atmel to become a global leader in multiple markets, and successfully diversify into these market areas. 
Atmel support the hypothesis of this report.

