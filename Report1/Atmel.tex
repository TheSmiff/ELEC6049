%  Atmel.tex
%  Document created by seblovett on seblovett-Ubuntu
%  Date created: Sun 16 Feb 2014 13:54:36 GMT
%  <+Last Edited: Sun 16 Feb 2014 17:01:45 GMT by seblovett on seblovett-Ubuntu +>

\subsect{Atmel Technologies Ltd.}

Atmel Technologies Ltd are a leading company in the design and manufacture of a variety of product areas. 
The company was founded in 1984 by George Perlegos in the USA, a former Intel employee. 
It started with only \$ 30,000 in capital \cite{atmel:capital} and made \$150.93 million profit in Q4 of 2013 alone \cite{atmel:profit}.
The company also has it's stocks in the NASDAQ \cite{atmel:nasdaq}.
Atmel satisfies the criteria for being an established company - it has been successful for many years, and has a defined product range.
This case study will give a brief overview of the company, including an investigation into the diversity of Atmel, and whether this diversity has helped the success of the company.



Atmel's original product range was in non-volatile memory devices, primarily EEPROM memory.
Atmel sold it's memory devices to companies such as Nokia and Motorola.
However, in 1987, Intel sued Atmel.
Intel claimed the Atmel had infringed their patent, and Atmel did not fight the case \cite{atmel:intel}.
Instead of spending time and money on the law suit, Atmel redesigned their memory devices and came up with a new design which turned out to have better performance and consumed less power than the Intel equivalent.
As well as beating Intel in the EEPROM market, Atmel entered the flash memory market and out competed Intel there too.
The patent case forced Atmel to redesign it's product, and can be seen as a, albeit small, diversification.
This resulted in a much better product than Intel, giving Atmel the edge in the market.


Since entering the memory market, Atmel also began to branch out to other markets. 
Atmel licensed an architecture from ARM to begin their microcontroller industry.
Since then, Atmel opened a branch in Norway to develop the AVR RISC architecture.
The company then combined their memory devices, both flash and EEPROM, with their AVR devices and the first microcontroller was released in 1996.

Not only did Atmel branch into relevant markets, but the company also acquired many other companies during their growth. 
They also bought a number of foundries from other companies, and later sold the majority of them.
Not only did the company gain different product areas, but they also sold parts of the company over time too. 
This raises the argument that too much diversification could be bad.
All acquisitions and sales are outlined below in chronological order.% and were found from \cite{atmel:acq1, atmel:acq2}
\inote{Maybe find some references for the earlier parts of this - but difficult to find a credible source.}
\begin{enumerate}
\item[1989] Purchased a Fab in Honeywell - enabled Atmel to conduct more R\&D %\cite{atmel:acq:}
\item[1991] Acquired Concurrent Logic - an FPGA manufacturer. %\cite{atmel:acq:}
\item[1994] Acquired Seeq Technologies - data communications and semiconductor devices%\cite{atmel:acq:}
\item[1996] Acquired European Silicon Structures and Digital Research in Electronic Acoustics%\cite{atmel:acq:}
\item[1998] purchased a part of TEMIC, a discrete semiconductor manufacturer. Included two fabs in France and Germany %\cite{atmel:acq:}
\item[2000] Purchased a fab in North Tyneside from Siemens %\cite{atmel:acq:}
\item[2005] Sold fab in Nantes, France \cite{atmel:acq:nantes05}
\item[2006] Sold fab in Grenoble, France \cite{atmel:acq:grenoble06}
\item[2007] Sold fab in North Tyneside, UK\cite{atmel:acq:tyneside07}
\item[2008] Sold fab in Heilbronn, Germany\cite{atmel:acq:heilbronn08}
\item[2008] Acquired Quantum Research Group to enter touch screen industry\cite{atmel:acq:qrg08}
\item[2010] Sold fab and branch in Rousset, France\cite{atmel:acq:sms10}
\item[2011] Sold D.R.E.A.M. %\cite{atmel:acq:}
\item[2011] Acquired Advanced Digital Design S.A who specialised in communications with power lines \cite{atmel:acq:add11}
\item[2012] Sold `DataFlash' products \cite{atmel:acq:sflash12}
\item[2012] Acquired Ozmo Devices who made Wi-Fi products %\cite{atmel:acq:}
\item[2013] Acquired Smart Metering products from IDT Corporation \cite{atmel:acq:smartmeter13}
\end{enumerate}

The acquisition and sales history of the company shows how diverse it it.
Atmel began in the memory market and moved to microcontrollers. 
At this time, it made sense to buy fabrication plants to aid the development of these areas. 
This was good diversification, and spanned Atmel's market into new, but relevant areas.
However, later on, Atmel sold almost all of it's fabrication houses. 
The company saw that making devices was not a good way to spend their money. 
Atmel now design microcontrollers and other related devices, testing them by using their remaining foundry, and getting them manufactured for the market by external companies.

Overall, Atmel have expanded their business into many different markets, including touch screens, automotive, Wi-Fi and security. 
They had a shaky start to life, but since have become a global leader in multiple areas of industry. 
Atmel support the hypothesis of this report.

