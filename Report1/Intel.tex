%  Intel.tex

\subsect{Intel Corporation}

Intel is a pioneering company in the field of microprocessor design and fabrication.
The company was founded in 1968 by Bob Noyce and Gordon Moore~\cite{IntelHistory}; the man behind the self fulfilling prophecy known as Moore's law - "The number of transistors incorporated in a chip will approximately double every 24 months."~\cite{IntelMoore}.
The first products intel produced were memory chips, including the 1101 and 1103 DRAM chips \cite{IntelStartup}.
Following the success in the memory market, Intel released the 4004, a four-bit microprocessor.
This was the first commercially available microprocessor and was released in November 1971 \cite{IntelStartup}.
Although the technology was emerging, the founders were already well established in the electronics industry.
Members of the "The Traitorous Eight" created the west coast silicon industry from which direct links can be made to many well established companies~\cite{Eight}; including Atmel.
The two experienced engineers had already made a mark in Fairchild semiconductor~\cite{Fairchild} and were backed by a \$2.5 million investment from Aurthur Rock~\cite{IntelStartup}.

A case study taken from \cite{johnson2008exploring} considers Intel as moving through three eras.
The first was involvement in the memory market followed by the second with huge successes in microprocessors.
The third era, possibly still their current drive, is moving focus away from the desktop market towards low power devices for the mobile technology market and the Internet of Things.
The end of the second era coincides with their highest every share price at the beginning of the 21$^{st}$ century then dropping rapidly shortly after and remaining below 50\% of the peak value up until this day~\cite{IntelStock}.

While Intel remains a profitable entity, the company failed to continue to profit as extensively as it did at the end of the second era. 
What is the reason for this relative failure?
Lack of diversification is a possible factor as the company may have failed to do this early enough.
Intel had particular expertise in microprocessor design and manufacture, and dominated the expanding PC CPU industry during the second era. The nineties saw ten years of 30\% per annum growth in the market, which only started to slow around 1998 \cite{johnson2008exploring}.
The company remained relatively specialist and failed to identify and capitalise on changes in the market.
Competitors in the processor market, namely ARM, came out of nowhere with innovative new business models and technology.
Their low power designs were suitable for the emerging portable digital electronics revolution on the horizon.
Intel needed to make some considerable corporate changes in technology and ethos to enter this new market.
Additionally, the mobile market poses a threat to the profitability of the desktop market Intel was already dominant in, and therefore marks a significant conflict of interest.
Although Intel are now adapting to these changing market conditions with their new Intel Atom product line and other low-power devices, the company remains on the backfoot in the market due to their lack of related diversification.
