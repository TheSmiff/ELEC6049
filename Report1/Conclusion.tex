\phantomsection
\addcontentsline{toc}{section}{Conclusions}
\sect{Conclusions}

All the companies studied in this report meets the requirements of an established company - they have been operating for a number of years with a defined product range \cite{kekale2007successful}.
There are clearly some common ideas throughout these case studies that have made a significant contribution to the success of each company.
All six of the considered companies emerge with a technical innovation, breakthrough or specific expertise.
Atmel developed a better memory device, ARM were experts in low-power CPU design, Domino formed on the back of a breakthrough in printing technology, Google formed following the page rank algorithm breakthrough, Imagination were experts in video hardware design and Intel designed the world's first commercial microprocessor.
Clearly, the initial competencies of the business are crucial to the company becoming established in the first place.
A successful company cannot be built on inferior technology, a flawed product or poor strategy no matter how diverse that company becomes.

When a company becomes established, it is sensible to asses why that success has happened.
Identifying the core competence of a business is of key strategic importance. 
Google successfully identified their core competence in advertising, Domino in printing technology and ARM in CPU architecture design. 
Atmel misdiagnosed it's core competence in it's foray into the fabrication industry, later correctly identifying hardware design as their core competence.
By understanding what it is about a company that makes it successful, the company can capitalise on these strengths to make further profit.

Related diversification into areas where the core competence of a business is advantageously applied allows an established company access to a new market and a new source of revenue.
This is a strategy employed by many of the companies that have been studied in this report.
Related diversification builds on the strengths of the company to gain a strategic advantage over their new competitors through the reapplication of core competencies.
The cases studies also highlight the dangers of not diversifying at all - as in the case of Intel missing an opportunity to profit in the mobile revolution, and also diversifying into unrelated areas, as shown by Atmel entering manufacturing.
Overall, there is a strong body of evidence to suggest that related diversification is an important strategy that established companies use to profit and continue to maximise their profit from technology.


%Some more stuff here that all these companies used related diversification in that it was related to their core competence as a business and this is what makes them successful. Trying to compare and contrast these case studies more to back-up the case.

%%% TJS - Feel free to run with this a bit more when you have a look AJR- I'll have another look in the morning. I liked what you were saying but I think we need to make the conclusion about drawing this all together into a coherent argument. However as I say go mad with it and see what you come up with.

%All with defined product ranges for specific customers apart from Google once again which has such as an extremely broad customer base.
%Is it possible Google's immediate success is down to extreme related diversification?
%The only unique idea they brought to field of search engines was their page rank algorithm.
%The search engine Yandex~\cite{Yandex} was already established before Google and even today still outperforms them in Russia. 
