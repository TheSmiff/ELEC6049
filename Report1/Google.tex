%  Google.tex

\subsect{Google Inc.}
Google Inc. was founded in 1998 \cite{google:timeline} by Larry Page and Sergey Brin based on their work carried out at Stanford University \cite{brin1998anatomy}. From humble beginnings as a search engine provider, Google has grown to dominate 70\% of the online search and advertisement industry \cite{rothrmel2013strategic} and has interests in a wide field of software applications \cite{wheelen2012strategic}. Google is one of the world's most valuable companies, with a market capital value of \$212.4bn \cite{FT500}. Google has successfully moved into areas as diverse as email, office software, geographical software, operating systems, mobile hardware and intelligent systems \cite{rothrmel2013strategic, wheelen2012strategic}. The success of these ventures can be attributed to consistently outperforming competitors with their core internet search services \cite{wheelen2012strategic}, which allows diversification to be based upon a renewal and reapplication of well-proven concepts.

Google's entry to the mobile operating system (OS) market in 2007 has been a worldwide phenomenon, with over 1 billion activated Android devices and 975,000 apps in the Android ecosystem \cite{google:timeline}. 79\% of all smartphones run the Android OS \cite{krajci2013android}. While the relationship between internet search and mobile OSs is not immediately apparent, the success of Android over it's competitors is through the re-application of the experience gained from the internet search industry. This includes all aspects of the Google approach, from the working environment to the corporate strategy. 

The way Google makes money is through advertising. The use of advertising as a money making method has been long perfected by google through it's core internet search service. Although there are other Android revenue streams such as royalties and charges for content from the ``play store'' the bulk of Google's profit from Android comes from advertisements \cite{krajci2013android}. The OS itself can be offered for free to the end user in much the same way the internet search service is free to use by the end user. This is a marked removal from the approaches of competitors such as Apple and Microsoft \cite{rothrmel2013strategic}, and allows Android to be installed on a range of phones from budget to flagship models. In 2013, there were over 3,900 differnet android device models \cite{krajci2013android} compared to just a few iPhone models. This drives Google's profits, not from the value of the individual devices sold but due to the greater reach of advertisements afforded by the larger user base.  

While not all companies that use related diversification necessarily succeed \cite{johnson2008exploring}, it has worked for Google. Google has been able to critically assess the areas of core competence gained from the internet search market, and identify other markets that could be exploited through these competencies. The amount of data Google is willingly given and the everyday use of Google services in everyday life makes Google very powerful, and we can certainly expect the Google model to be used in more and more market scenarios.

