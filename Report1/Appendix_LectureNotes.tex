%  Appendix_LectureNotes.tex
%  Document created by seblovett on seblovett-Ubuntu
%  Date created: Sun 16 Feb 2014 09:34:23 GMT
%  <+Last Edited: Sun 16 Feb 2014 09:35:47 GMT by seblovett on seblovett-Ubuntu +>


\sect{4. Notes from Invited Presentations} \label{Notes}
These notes are raw and not altered in any way from when they were taken from the invited presentation. These notes have been distilled and focussed through the lens of our report title and hypothesis to the content shown in section 1.
\subsect{Domino Printing Sciences PLC - Carl Reynaud - Director of Hardware Development}
Success - ``bringing new products into market and making a profit.''
The key question for any business is how to make profit in a sustainable manner. You need to be able to make profit, then keep on making a profit.

Background to Domino Printing - mission statement - we will achieve market recognition as the first choice global provider of coding, marketing and variable printing solutions delivering convenience, security and peace of mind. Key markets for Domino include;
\begin{itemize}
\item Date coding packaging - Slippery surfaces requires special technology.
\item Industrial printing and coding - this is on fast moving, small and hot plastics and requires a similar technology to date coding.
\item Printing and Mailing - Completely different technology required to the other two market areas.
\end{itemize}

Domino started with a key breakthrough technology in inkjet printing. This has led to diversification into laser and multijet technologies. The head office is in Cambridge. Domino manufacture around 1000-10,000 products per year, making it a low-medium volume business. This presents a number of challenges and opportunities. The technology portfolio ranges from ideas to mature technologies.

``Most ideas take 20 years to become an overnight success'' - Paul Saffo.

Four key points:
\begin{itemize}
\item You want what?
\item Celebrating Diversity
\item No absolutes
\item Keep updating the toolbox
\end{itemize}

You want what highlights that people are often the forgotten ingredient. It is key to know your customer. The customer often lacks the ability to articulate the problem properly and also lacks the vision to see the solutions. Identify what the customer really values (not necessarily the same as what they say they value) and why that is the case, as that is what customers will pay for. As an engineer, just because it is possible doesn't mean we should do it. Does the world need a better mouse trap? There is a significant skill in being able to capture and distill customer values and then apply that to a practical concept. A practical example was given of a customer specification asking for a printer weighing less than 25kg. Asking why this was the case broke the specification down to a real world requirement, that the printer should be able to be installed by one person to minimise installation costs. This requirement requires far more consideration than just weight and concerns shape, handling, regulation (lone working) and other factors. This is called a 5Y process. Another useful model to follow is the V-model.

Celebrating Diversity - don't try to put in what nature left out. changes in culture, education adn psychology. We need to identify how to work best with each other as a team. This involves knowing how you personally work, and identifying how others you work with behave so that you make best use of the diverse skillset in your team. Meyers Briggs type indicators are often used here. Diversity is particularly important when considering foreign markets. 

No absolutes - there will always be variations, but what is the failure zone. Process capability - understand variance. Need to identify suitable margins of error. Test products to make them fail - don't let the customer have a failed product as that can be disasterous. 

Keep updating the toolbox - Pareto and the 80:20 rule was mentioned, identify the 20\% of the work that delivers 80\% of the value etc. Failure mode effect and analysis provides a framework for risk analysis. 

%\inote{Questions and Answers (Ashley)}

Question and answer session:

"What's the biggest barrier you have encountered as an engineer?" - 
Not understanding how to apply technology as such to make it valuable for the customer. 

"How do the features of an engineering product relate to its success?" - 
Good products tend to have less features. 
It may be the case that less time can be spent fixing a parameter by researching customer preferences than implementing parameter adjusting functionality.
For example take the iPad versus a laptop.
Older generations will choose an iPad because there are less features therefore making it simpler to operate.

"Does Domino have any plans for expansion?" - 
We are currently invested in date code printing but are making a move to variable surface printing and also the printing of different materials.

"You mentioned how knowing your customer was important. How could a business suceed with emrging technologies when no record of the customer is already known?" -
There may exist parallel markets which can give some confidence of an expected customer base.
Focus groups and trade shows are a good way to test new technologies without fully deploying any products.
Constantly requesting customer input is also a good way to update the possible requirements which would lead to the design of a sucessful product.

"How do you measure success?" -
I look at technical forums where customers seek solutions to problems.

"Does domino have any plans to takeover smaller businesses?" -
Not at the present.
We tend to look for businesses that are making a profit rather the businesses which are producing future possible technologies.

"Have you thought of diversifying by making components you would otherwise outsource?" - 
Some of our print heads are brought in because the piezo electronics is too complicated for us to invest resources.
The mounting mechanisms however are designed internally but built externally. 
This allows to retain the intellectual property but not get involved with the machining required for implementation.
We look internally for skills that already exist onto which we could diversify rather than jumping to a completely new business. 

%%%%%%%%%%%%%%%%%%%%%%%%%%%%%%%%%%%%%%%%%%%%%%%%%%%%%%%%%%%%%
%% ARM LECTURE NOTES
%%%%%%%%%%%%%%%%%%%%%%%%%%%%%%%%%%%%%%%%%%%%%%%%%%%%%%%%%%%%%
\subsect{ARM - John Biggs - Senior Engineer}
John has been at ARM since 1986 and helped form ARM in 1990. 
ARM - The architecture of a digital world.
ARM is the worlds leading semiconductor IP company with 30 million processors entering the market every day.
Over 50 billion ARM chips have been shipped to date, 10 billion of which in 2013. 
A major driver is the mobile revolution, with smartphones and tablets vastly outselling desktop machines.

Acorn was founded on 5th December 1978 by Hermann Hauser and Chris Curry. 
Their first contract was to develop fruit machine hardware, to replace the conventional electromechanical solutions. 
The Acorn system 1 was sold for 70 pounds in 1979 based on the 8 bit 6502 processor. 
Acorn's next big hit was the BBC Micro in 1982. 
Chris Curry told the BBC that they would have a working prototype within a week, and the Acorn team just about managed to build it in time. 
Having won the BBC Micro contract, Acorn went on to make a number of other 6502 based machines, including the Electron and Master.

Acorn needed a more powerful computer, and looked to Intel to license the 80286 processor, which was refused. 
As a result Acorn's advanced R\&D labs was set up to build Acorn's own 32-bit processor. 
In 1983, Acorn engineers were inspired by a trip to the Western Design Centre. 
The small scale of this operation gave Acorn confidence to design their own chip. 
Hermann gave the design team two key advantages. 
There was no money, and no people for the project meaning the design had to be simple and elegant.

The first ARM silicon was built in 1985 with 3 micron technology, 25k transistors, 6MHz and less than 0.1W power. 
This was the worlds first commercial available RISC processor. 

To reduce the cost of a home computer, Acorn continued to design further chips, memory controller, video controller and IO controller which were assembled onto a single board. 
These were eventually launched as the Archimedes in 1997.

PDAs started to emerge around 1988 based around portability and handwriting recognition.
Low power chips were required, and Hermann asked ARM to develop an especially low power chip. 
Around the same time, Apple were developing the Newton Message Pad based around the AT\&T Hobbit processor.
In the end they swapped it for the ARM chip, using the ARM610. 
In a bizarre twist of fate the machine designed by Hermann and the Apple device swapped processor. 
The ARM610 was made from 1.2 micron technology and some 358 thousand transistors.

In 1997 ARM was founded as a joint venture with Acorn and Apple, since Apple required to have control over their IP. 
Apple put in 1.5 million pound capital, and Acorn supplied the people and the IP (valued similarly). 
Acorn was financially troubled during the 1980s, and were trying to sell the research group any way. 
So the ARM spin out was mutually agreeable. 
Robin Saxby was decided to be the CEO of the ARM spin out, after meeting him in a pub. 

ARM took some of the benefits of RISC architectures and attempt to reduce costs. 
Robin Saxby was initially advised not to work for ARM as "joint ventures never work". 
He performed a simple SWOT analysis and identified some of the key areas of value in ARM. 
Robin took some of the engineers and promoted them to commercial roles. 
The early years were not easy, 92 featured a company wide pay freeze. 

Robin Saxby - \emph{"If you aren't making mistakes you aren't trying hard enough"}.

Robin developed the partnership model. 
Did he grow ARM until it was acquired by a larger company, grow to become a semiconductor company in its own right or become more embedded with Apple? 
Robin did none of these, and developed the licensee/royalty partnership model. 

The first licensee was with GEC and Sharp. 
GEC were in Portsmouth, and Sharp in Japan which meant they were geographically separate. 
Next the automotive part of TI and Samsung also became a licensee. 
By 1995 ARM started to gain international recognition due to this model, despite there was only 40 people actually working for ARM. 
It was the leverage provided by the partners that made the difference.

The early licenses were perpetual licenses. 
As time progressed, further license terms were developed. 
Term licenses allowed a time limited version of the perpetual license, and there were also several other types. 

1993 - ARM was 3 years old when Nokia approached TI to build a new chipset based on the ARM 7. 
But the footprint was too large due to large 32-bit instruction set. 
The ARM7TDMI was built with a high code density for the mobile phone market. 
This had 170 licensees and shipped over 10 billion of these.
It took until 1996 for the product to enter the market, the Nokia 8110.
Innovative layout techniques were developed which helped this processor be so successful.

In 1998 ARM was a 27 million pound business with a net income of 3 million pounds. 
It was time to float the company in April 1998. 
The stocks soared, and ARM became a billion dollar company overnight. 
The acronym was also dropped at this point.

IP deployment has 3 players, the creator, implementor and integrator. 
A key change in the process was handing over soft IP, so that the implementor developed the processor from RTL to a final design. 
This was driven by the consumer. Developments in logic synthesis meant that soft IP was becoming favorable in terms of flexibility and time to market.
ARM developed a reference methodology to minimise the effort to integrate this soft IP into their product, depending on the required features. 
ARM needed to completely change their technology in order to take advantage of synthesisable cores. 
This became the foundation for the ARM9 and others.

In 2000 the ARM926E was developed which is one of ARMs most successful processors selling over 5bn units. 
The ARM instruction set has had to develop considerably to allow some of the more advanced and extended features. 
Eventually the ARM processors were divided into three families, focusing on different aspects.

2008 saw the ARM Cortex-A9 which was a step forward in multicore processing. 
This was driven by consumers wanting ever more advanced user experiences in mobile devices but also with longer battery life. 
2011 saw the Cortex-A15 which developed the idea of big-little connected by a coherent cache. 
This allowed energy savings of approximately 70\%. 

Looking to the future, a new generation of computing is set to take hold. 
Currently we are in the Mobile Internet phase, but soon entering The Internet of Things phase. 
The Cortex-M0 was built in 2012 which is the most energy efficient 32-bit processor ever built. 
It is aimed at low power IoT applications, such as wireless sensor nodes etc. Much more compact design. 
Freescale produced a processor based on this technology that measures just 2mm square, for ideas such as ingestible electronics.

ARM and ECS at Southampton started a joint research project in 2008.
 Mostly the work is based in the area of energy efficiency. 
The first PhD student has just graduated.

John then made some reflections on how things have changed at ARM. 
Synthesis tools have improved implementation time from 6 months for the ARM1 to just 30 minutes with the Cortex-M0. 
This is in-keeping with Moore's Law. 
Area scaling follows similarly. 
Performance and Voltage do not follow Moore's Law due to the laws of physics. 
However, Power Efficiency has also managed to be increased with Moore's Law. 
ARM is also now a much more global company and is much more connected. 
A couple of extreme examples is a $1 mm^2$ implementation of the Cortex-M0. 
At the other end is a $1 km^3$ computer for neutrino monitoring in the Arctic. 
There is a huge diversity in application. 

ARM servers are now increasingly important, and represent a huge opportunity for ARM. 
ARM's partnership model is paying off in this area as companies seek to produce micro servers that save huge amounts of space, cooling and energy. 
This saves around half the cost of a conventional data sensor.

A world where all electronic products and services are based upon energy efficient technology from ARM, making life better for everyone.

Key lessons:
\begin{itemize}
\item Top-right isn't everything
\item Design once, use many times
\item The partnership is everything
\item Listen to your customer, and their customer
\item Timescales are long
\item People are the biggest asset we have
\item It pays to be different
\item Strive for simplicity beyond complexity.
\end{itemize}

Alan Perlis - \emph{``Fools ignore complexity; pragmatists suffer it; experts avoid it; geniuses remove it!''}

Questions:
Why so many offices - culturally it is better to go and meet the customer face to  face.
Is it beneficial to invest in companies attributed to ARM - No, focusing on licensing and royalty model.
Is the ARM-ECS relationship important - increasingly so. ARM investing in lots of areas, science museums and Raspberry Pi as corporate social responsibility.
What point did ARM feel established and secure in the market - 1996 a guy from Samsung asked if he could come and speak to ARM partners, and he gave the motivational speech.
How important is the mobile sector - It was integral to the business, but now ARM is in so many different market areas.
What makes ARM unique - The partnership model is critical. Defends against aggressive take over from big players like Intel.
Are ARM considering buying any other companies - yes ARM bought Mali graphics just a few years ago that compete heavily with imagination technologies.
How do ARM deal with competing with Imagination and Mali - Issues are more to do with Synopsis as you need synopsis to build these soft cores, but they also have their own IP which can make working together difficult at times.
