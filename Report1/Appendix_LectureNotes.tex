%  Appendix_LectureNotes.tex
%  Document created by seblovett on seblovett-Ubuntu
%  Date created: Sun 16 Feb 2014 09:34:23 GMT
%  <+Last Edited: Sun 16 Feb 2014 09:35:47 GMT by seblovett on seblovett-Ubuntu +>


\sect{4. Notes from Invited Presentations} \label{Notes}
These notes are raw and not altered in any way from when they were taken from the invited presentation. These notes have been distilled and focussed through the lens of our report title and hypothesis to the content shown in section 1.
\subsect{Domino Printing Sciences PLC - Carl Reynaud - Director of Hardware Development}
Success - ``bringing new products into market and making a profit.''
The key question for any business is how to make profit in a sustainable manner. You need to be able to make profit, then keep on making a profit.

Background to Domino Printing - mission statement - we will achieve market recognition as the first choice global provider of coding, marketing and variable printing solutions delivering convenience, security and peace of mind. Key markets for Domino include;
\begin{itemize}
\item Date coding packaging - Slippery surfaces requires special technology.
\item Industrial printing and coding - this is on fast moving, small and hot plastics and requires a similar technology to date coding.
\item Printing and Mailing - Completely different technology required to the other two market areas.
\end{itemize}

Domino started with a key breakthrough technology in inkjet printing. This has led to diversification into laser and multijet technologies. The head office is in Cambridge. Domino manufacture around 1000-10,000 products per year, making it a low-medium volume business. This presents a number of challenges and opportunities. The technology portfolio ranges from ideas to mature technologies.

``Most ideas take 20 years to become an overnight success'' - Paul Saffo.

Four key points:
\begin{itemize}
\item You want what?
\item Celebrating Diversity
\item No absolutes
\item Keep updating the toolbox
\end{itemize}

You want what highlights that people are often the forgotten ingredient. It is key to know your customer. The customer often lacks the ability to articulate the problem properly and also lacks the vision to see the solutions. Identify what the customer really values (not necessarily the same as what they say they value) and why that is the case, as that is what customers will pay for. As an engineer, just because it is possible doesn't mean we should do it. Does the world need a better mouse trap? There is a significant skill in being able to capture and distill customer values and then apply that to a practical concept. A practical example was given of a customer specification asking for a printer weighing less than 25kg. Asking why this was the case broke the specification down to a real world requirement, that the printer should be able to be installed by one person to minimise installation costs. This requirement requires far more consideration than just weight and concerns shape, handling, regulation (lone working) and other factors. This is called a 5Y process. Another useful model to follow is the V-model.

Celebrating Diversity - don't try to put in what nature left out. changes in culture, education adn psychology. We need to identify how to work best with each other as a team. This involves knowing how you personally work, and identifying how others you work with behave so that you make best use of the diverse skillset in your team. Meyers Briggs type indicators are often used here. Diversity is particularly important when considering foreign markets. 

No absolutes - there will always be variations, but what is the failure zone. Process capability - understand variance. Need to identify suitable margins of error. Test products to make them fail - don't let the customer have a failed product as that can be disasterous. 

Keep updating the toolbox - Pareto and the 80:20 rule was mentioned, identify the 20\% of the work that delivers 80\% of the value etc. Failure mode effect and analysis provides a framework for risk analysis. 

%\inote{Questions and Answers (Ashley)}

Question and answer session:

"What's the biggest barrier you have encountered as an engineer?" - 
Not understanding how to apply technology as such to make it valuable for the customer. 

"How do the features of an engineering product relate to its success?" - 
Good products tend to have less features. 
It may be the case that less time can be spent fixing a parameter by researching customer preferences than implementing parameter adjusting functionality.
For example take the iPad versus a laptop.
Older generations will choose an iPad because there are less features therefore making it simpler to operate.

"Does Domino have any plans for expansion?" - 
We are currently invested in date code printing but are making a move to variable surface printing and also the printing of different materials.

"You mentioned how knowing your customer was important. How could a business suceed with emrging technologies when no record of the customer is already known?" -
There may exist parallel markets which can give some confidence of an expected customer base.
Focus groups and trade shows are a good way to test new technologies without fully deploying any products.
Constantly requesting customer input is also a good way to update the possible requirements which would lead to the design of a sucessful product.

"How do you measure success?" -
I look at technical forums where customers seek solutions to problems.

"Does domino have any plans to takeover smaller businesses?" -
Not at the present.
We tend to look for businesses that are making a profit rather the businesses which are producing future possible technologies.

"Have you thought of diversifying by making components you would otherwise outsource?" - 
Some of our print heads are brought in because the piezo electronics is too complicated for us to invest resources.
The mounting mechanisms however are designed internally but built externally. 
This allows to retain the intellectual property but not get involved with the machining required for implementation.
We look internally for skills that already exist onto which we could diversify rather than jumping to a completely new business. 
