%  Appendix_Minutes.tex
%  Document created by seblovett on seblovett-Ubuntu
%  Date created: Sun 16 Feb 2014 09:32:52 GMT
%  <+Last Edited: Sun 16 Feb 2014 09:33:34 GMT by seblovett on seblovett-Ubuntu +>
\phantomsection
\addcontentsline{toc}{section}{Appendix 2 - Minutes from Kick-off Meeting}
\sect{2. Meeting Minutes - Kick-off Meeting}
\begin{center}
\begin{longtable}{| m{0.2\textwidth} | m{0.6\textwidth} |} \hline
\textbf{Purpose} & ELEC6049 Team Kick-Off Meeting \\ \hline
\textbf{Date and Time} & Tuesday 4th February 2014 at 12:00 \\ \hline
\textbf{Venue} & 3rd floor Zepler Building, Highfield Campus \\ \hline
\textbf{Participants} & TJS (Tom Smith), HSL (Henry Lovett), AJR (Ashley Robinson)\\ \hline
\textbf{Apologies} &None \\ \hline
\multirow{4}{*}{\textbf{Agenda}} & Assign Chair for this report. \\
 & Generate initial ideas for research. \\ 
 & Agree expectations of work and schedule. \\
 & Agree date and agenda of next meeting. \\ \hline
\end{longtable}
\end{center}

\subsect{Minutes of the Meeting}
\begin{center}
\begin{longtable}{| p{0.05\textwidth} |>{\raggedright\arraybackslash}p{0.15\textwidth} | p{0.5\textwidth} |>{\raggedright\arraybackslash}p{0.175\textwidth}|} \hline
\textbf{ID} & \textbf{Subject} & \textbf{Notes and Discussion} & \textbf{Action} \\ \hline
\endhead
1.0	&	Chair	&	The group decided that TJS should be the chair for this report. Both HSL and AJR are currently pursuing job applications, hence the logical choice that TJS be chair for this report. 	&   -	 \\ \hline
2.0	&	Case Study Ideas	&	TJS presented some ideas previously undertaken in a management module regarding Bowman's Strategy Clock, and the different strategies employed in the tablet market. The group discussed this as a case study and agreed it was interesting but obvious, but struggled to make a hypothesis. Discussions of reputation (Apple) and quality.	&-	 \\ \hline
3.0	&	Hypothesis 1	& HSL discussed Reputation - Companies such as Intel, Apple, Microsoft rely on their reputation as an established brand to sell their products. AJR proposed quality along the same lines.  & - \\ \hline
3.1	&	Hypothesis 2	& The group discussed collaboration in the context of ARM being fabless which enabled it to be more agile than Intel in the microprocessor market. AJR highlighted that ARM and Xilinix collaborate in the microprocessor and FPGA markets. Hypothesis is that by collaborating, established companies make money by reducing and spreading risks across multiple entities. Apple is an example contradicting the hypothesis. &	- \\ \hline
3.2	& Hypothesis 3 & TJS proposed that smart diversification was how an established company makes money. The group discussed Apple as an example with the iPhone, iPad and Mac ranges. Also Google with the search engine at the core, and Chrome and Android as diversification. Amazon with the online store at the core, and the Kindle e-reader and tablet as smart diversification along the same theme. A quick look in a textbook shows that this is called ``Related Diversification''. Microsoft with OSs and XBox. The group agreed that this was a possible hypothesis. &- \\ \hline
3.3 & Hypothesis selection & The group narrowed it down to collaboration or diversification as a topic. TJS chose to head toward the diversification hypothesis due to the strength and number of case studies. & \textbf{ALL A1.0} \textbf{All A3.0} \\ \hline
4.0 & Milestones & The group agreed to meet again on the \textbf{11th Feb} and that by then, most research should be complete, report stubs should be started and a clear hypothesis should be sought. The next meeting will be to put what we have together into a plan and to allocate work to complete the report by the tutorial on \textbf{18th Feb}. & \textbf{TJS A2.0} \\ \hline


\end{longtable}
\end{center}

\subsect{Action List}
\begin{center}
\begin{longtable}{| p{0.05\textwidth} | >{\raggedright\arraybackslash}p{0.15\textwidth} |  p{0.5\textwidth} | >{\raggedright\arraybackslash}p{0.175\textwidth}|} \hline
\textbf{ID} & \textbf{Action} & \textbf{Comments} & \textbf{Status} \\ \hline
\endhead
A1.0	&	Research	&	All to start research. Use Git Issue to highlight useful research to the group. Make notes of all sources.	& Open 4th Feb \\ \hline
A2.0	&	Introduction	&	Define an established company. Introduce hypothesis.	&	Open 4th Feb \\ \hline
A3.0	&	Case Study	&	Each to identify a case study for proposal to the group by next week.	&	Open 4th Feb	\\ \hline	
\end{longtable}
\end{center}

\subsect{Next Meeting: 11th Feb 2014, 12:00, Level 3 Zepler, Highfield Campus.}

%\clearpage

\phantomsection
\addcontentsline{toc}{section}{Appendix 3 - Minutes from Progress Meeting}
\sect{3. Meeting Minutes - Progress Meeting}
\begin{center}
\begin{longtable}{| m{0.2\textwidth} | m{0.6\textwidth} |} \hline
\textbf{Purpose} & ELEC6049 Report Meeting \\ \hline
\textbf{Date and Time} & Wednesday 12th February at 11:00 \\ \hline
\textbf{Venue} & 3rd floor Zepler Building, Highfield Campus \\ \hline
\textbf{Participants} & TJS (Tom Smith), HSL (Henry Lovett), AJR (Ashley Robinson)\\ \hline
\textbf{Apologies} & None \\ \hline
\multirow{5}{*}{\textbf{Agenda}} & Update on research status. \\
 & Discuss case study examples. \\ 
 & Read any report stubs as a group. \\
 & Identify and allocate work to finish report. \\ 
 & Agree next meeting. \\ \hline
\end{longtable}
\end{center}

\subsect{Minutes of the Meeting}
\begin{center}
\begin{longtable}{| p{0.05\textwidth} |>{\raggedright\arraybackslash}p{0.15\textwidth} | p{0.5\textwidth} |>{\raggedright\arraybackslash}p{0.175\textwidth}|} \hline
\textbf{ID} & \textbf{Subject} & \textbf{Notes and Discussion} & \textbf{Action} \\ \hline
\endhead
1.0	&	House Keeping	&	HSL identified and addendum to the minutes of the last meeting and has corrected the typo. 
						Subsequent to this the minutes of the last meeting were adopted by the group								&	- \\ \hline
2.0	&	Tutorial Session	&	TJS attended the non-official tutorial session with Chris Freeman (CF). 
						TJS fed back to the group that CF had been very positive regarding the hypothesis proposed. 
						CF said it looked like a good report with good examples to back it up, the ideas were received with considerable enthusiasm. 
						CF highlighted the importance of good and appropriate referencing which should be heeded.						&	- \\ \hline
3.0	&	Research Progress	&	HSL provided some initial reading on Google, Amazon and Apple with particular emphasis on ZDnet article.
						AJR questioned if this is a credible source. 
						The group agreed it is important to follow reference trails through literature to establish the original source.			 &	- \\ \hline
4.0	&	Report Progress	&	TJS has drafted an introduction to the report and written the Domino Printing section. 
						The group read the report so far. 														 & 	   \\
4.1     &        				&	HSL noted that IEEE allows you to refer to numbered references as nouns, which the group agreed should be followed despite the report not being explicitly IEEE style.  &	\textbf{TJS A4.0} \\ 
4.2	&				&	HSL suggested moving the introductions from the start of each section to the introduction of the report - Actioned.		&\\
4.3	&				&	TJS proposed that the Domino Printing case study should be used as a template - Agreed by the group.				&	   \\
4.4	&				&	The group decided on three or potentially four case studies for section 2. 
						Google was chosen as an example of related diversification. 
						It was mentioned in the previous meeting as an example. 
						Atmel was also suggested, and HSL has some insider experience in this area hence justifying the choice. 
						Intel were discussed by the group. 
						Their failure to diversify into mobile must have caused some damage. 
						This will make a good study of a company that has remained relatively specialised. 
						Finally another company on the other end of the spectrum will be chosen. 
						Virgin Group was discussed with operations in finance, travel, space exploration and just about everything. 
						Mitsibushi were also discussed, with unrelated diversification spanning power equipment, cars and nuclear power stations. 
						One of these conglomerate organisations will also form a comparison for section 2. 
						The group agreed that the case studies should be written by the next tutorial, so that then the report could be collated into a coherent document.			&  \textbf{TJS A5.0 \& A8.0, AJR A6.0, HSL A7.0} \\
4.5	&				&	HSL mentioned that the group should break up the \TeX files. 										&  \textbf{HSL A9.0}	 \\ \hline
5.0	&	Remaining Talks	& 	The team agreed that HSL and AJR will take responsibility for one of the remaining invited talks each. 
						This includes keeping lecture notes in the appendix of the report and writing the relevant subsection in section 1.			& \textbf{AJR A10.0, HSL A11.0} \\ \hline
6.0	&	Minutes		&	The group agreed that TJS would collate these minutes.											& \textbf{TJS A12.0} \\ \hline
7.0	&	AOB			&	The next meeting will take place at the tutorial. The meeting was closed. & - \\ \hline

\end{longtable}
\end{center}

\subsect{Action List}
\begin{center}
\begin{longtable}{| p{0.05\textwidth} | >{\raggedright\arraybackslash}p{0.15\textwidth} |  p{0.5\textwidth} | >{\raggedright\arraybackslash}p{0.175\textwidth}|} \hline
\textbf{ID} & \textbf{Action} & \textbf{Comments} & \textbf{Status} \\ \hline
\endhead
A1.0	&	Research	&	All to start research. Use Git Issue to highlight useful research to the group. Make notes of all sources.	& Open 4th Feb - Closed 11th Feb \\ \hline
A2.0	&	Introduction	&	Define an established company. Introduce hypothesis.	&	Open 4th Feb - Closed 11th Feb \\ \hline
A3.0	&	Case Study	&	Each to identify a case study for proposal to the group by next week.	&	Open 4th Feb - Closed 11th Feb	\\ \hline
A4.0   &	IEEE Styling	&	Change reference from author and citation to citation as a noun		&	Open 11th Feb	\\ \hline
A5.0	&	Google Case Study	&	Research Google as a successful implementation of related diversification.	&	Open 11th Feb	\\ \hline 
A6.0	&	Intel Case Study	&	Research Google as a specialised company and the effects of not diversifying.   &	Open 11th Feb	\\ \hline
A7.0	&	Atmel Case Study	&	Research Atmel as a successful implementation of related diversification.		&	Open 11th Feb	\\ \hline
A8.0	& 	Conglomerate Case Study	&	Research a conglomerate (Virgin, Mitsibushi etc) for an example of unrelated diversification. 	&	Open 11th Feb	\\ \hline
A9.0	&	\TeX files	&	Break up the \TeX files into sections so the \LaTeX word count will work properly. 	&	Open 11th Feb	\\ \hline
A10.0	&	ARM Talk	&	Write up the invited talk notes and corresponding Section 1 part.	&	Open 11th Feb	\\ \hline
A11.0	&	Imagination Talk 	&	Write up the invited talk notes and corresponding Section 1 part.	&	Open 11th Feb	\\ \hline
A12.0	&	Minutes	&	Write up the minutes and circulate. & Open 11th Feb \\ \hline
\end{longtable}
\end{center}

\subsect{Next Meeting: 18th Feb 14, in the scheduled tutorial session 12:00 05/2015.}
\phantomsection
\addcontentsline{toc}{section}{Appendix 4 - Mintutes from Final Meeting}
\sect{4. Meeting Minutes - Final Progress Meeting}
\begin{center}
\begin{longtable}{| m{0.2\textwidth} | m{0.6\textwidth} |} \hline
\textbf{Purpose} & ELEC6049 Report Meeting \\ \hline
\textbf{Date and Time} & Tuesday 18th February at 12:00 \\ \hline
\textbf{Venue} & Building 2, Highfield Campus \\ \hline
\textbf{Participants} & TJS (Tom Smith), HSL (Henry Lovett), AJR (Ashley Robinson)\\ \hline
\textbf{Apologies} & None \\ \hline
\multirow{5}{*}{\textbf{Agenda}} & Update on research status. \\
 & Discuss case study examples. \\ 
 & Read through report as it stands. \\
 & Identify and allocate work to finish report. \\ 
 & Agree next meeting. \\ \hline
\end{longtable}
\end{center}

\subsect{Minutes of the Meeting}
\begin{center}
\begin{longtable}{| p{0.05\textwidth} |>{\raggedright\arraybackslash}p{0.15\textwidth} | p{0.5\textwidth} |>{\raggedright\arraybackslash}p{0.175\textwidth}|} \hline
\textbf{ID} & \textbf{Subject} & \textbf{Notes and Discussion} & \textbf{Action} \\ \hline
\endhead
1.0	&	House Keeping	&	The group adopted the last meeting minutes.													&	- \\ \hline
2.0	&	Introduction		&	Identified a small change in the introduction that was changed during the meeting.							&	- \\ \hline
3.0	&	Action List Review	&	A1.0, A2.0 and A3.0 closed at last meeting													 & 	   \\
3.1	&				&	A4.0 Outstanding but is a very minor change. Actioned by TJS and closed.								 & 	   \\
3.2	&				&	A5.0 and A7.0 first draft achieved. To be reviewed in this meeting.									 & 	   \\
3.3	&				&	A6.0 not complete due to AJR interview and report commitments. To be completed shortly.						 & \textbf{AJR A6.0} 	   \\
3.4	&				&	A8.0 not sure if required due to word count and relevance.											 & 	   \\
3.5	&				&	A9.0 Done - actioned by HSL															 & 	   \\
3.6	&				&	A10.0 assigned to HSL and A11.0 assigned to AJR - swapped due to HSL interview commitments.					 &  \textbf{HSL A10.0 and AJR A11.0} \\
3.7	&				&	A12.0 TJS will write these minutes.														 &  \textbf{TJS A12.0} \\ \hline
4.0	&	Atmel Discussion	&	The group discussed the Atmel case study by HSL. 
						Generally agree the case study is good needs condensing slightly. 
						Focus on the points not necessarily the history although the history is necessary for understanding. 
						Drew parrallels perhaps between Intel with high stocks at one point then a change. 
						All to read it a bit and review, HSL to condense.												&  \textbf{ALL A7.0} \\ \hline
5.0	&	Google Discussion	&	The group discussed the Google case study by TJS. 
						Main point that Google had a core capability and reapplied that in a related sector.
						Group agreed this was good and looked at Intel as a company that was specialised and didn't reuse those skills.
						All to read it a bit and review.															&  \textbf{ALL A5.0} \\ \hline
6.0	&	Reference Format	&	Asked the tutorial leader what the reference format should be. 
						Not entirely important but need to ensure it is traceable.
						Need to make clearer. 																&  \textbf{HSL A13.0} \\ \hline
7.0	&	Outstanding Work	&	AJR to do Intel and Imagination and start conclusions. 
						HSL to do ARM and References. 
						TJS to do minutes, conclusion and comparisons.
						Abstract will be done last (TJS take ultimate responsibility as chair).									&  \textbf{AJR/TJS A14.0} \\ \hline
8.0	&	AOB			&	The next meeting will take place at the tutorial. The meeting was closed. & - \\ \hline

\end{longtable}
\end{center}

\subsect{Action List}
\begin{center}
\begin{longtable}{| p{0.05\textwidth} | >{\raggedright\arraybackslash}p{0.15\textwidth} |  p{0.5\textwidth} | >{\raggedright\arraybackslash}p{0.175\textwidth}|} \hline
\textbf{ID} & \textbf{Action} & \textbf{Comments} & \textbf{Status} \\ \hline
\endhead
A1.0	&	Research	&	All to start research. Use Git Issue to highlight useful research to the group. Make notes of all sources.	& Open 4th Feb - Closed 11th Feb \\ \hline
A2.0	&	Introduction	&	Define an established company. Introduce hypothesis.	&	Open 4th Feb - Closed 11th Feb \\ \hline
A3.0	&	Case Study	&	Each to identify a case study for proposal to the group by next week.	&	Open 4th Feb - Closed 11th Feb	\\ \hline
A4.0   &	IEEE Styling	&	Change reference from author and citation to citation as a noun		&	Open 11th Feb - Closed 11th Feb	\\ \hline
A5.0	&	Google Case Study	&	Review Google as a successful implementation of related diversification.	&	Open 11th Feb, Mod 18th Feb	\\ \hline 
A6.0	&	Intel Case Study	&	Research Intel as a specialised company and the effects of not diversifying.   &	Open 11th Feb, Mod 18th Feb	\\ \hline
A7.0	&	Atmel Case Study	&	Review Atmel as a successful implementation of related diversification.		&	Open 11th Feb	\\ \hline
A8.0	& 	Conglomerate Case Study	&	Research a conglomerate (Virgin, Mitsibushi etc) for an example of unrelated diversification. 	&	Open 11th Feb	\\ \hline
A9.0	&	\TeX files	&	Break up the \TeX files into sections so the \LaTeX word count will work properly. 	&	Open 11th Feb	\\ \hline
A10.0	&	ARM Talk	&	Write up the invited talk notes and corresponding Section 1 part.	&	Open 11th Feb	\\ \hline
A11.0	&	Imagination Talk 	&	Write up the invited talk notes and corresponding Section 1 part.	&	Open 11th Feb	\\ \hline
A12.0	&	Minutes	&	Write up the minutes and circulate. & Open 11th Feb \\ \hline
A13.0&	References	&	Format references better.	& Open 18th Feb \\ \hline
A14.0&	Conclusions and Abstract	&	Weave report together into one choerent document and finish.	& Open 18th Feb \\ \hline
\end{longtable}
\end{center}

\subsect{Next Meeting: None - Coordinate through GitHub Issues.}




