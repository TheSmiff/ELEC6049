%  Introduction.tex
%  Document created by seblovett on seblovett-Ubuntu
%  Date created: Sun 16 Feb 2014 09:18:38 GMT
%  <+Last Edited: Sun 16 Feb 2014 17:03:50 GMT by seblovett on seblovett-Ubuntu +>

\sect{Introduction}
%This should cover the general concepts and principles relevant to the report \cite{kekale2007successful}, explain constructive \cite{ansoff1957strategies}, \cite{johnson2008exploring} and destructive technologies, compliance or similar in the context of electrical/ electronic/electromechanical technologies [1]). The purpose of the introduction is to 'set the scene'.  

In the context of this report, the definition of an established company is inspired by Kek{\"a}le et al \cite{kekale2007successful}. 
Established companies are those that have been successful for a number of years, with a defined product range for a specific consumer market. 
As a company develops and expands, a clear corporate strategy is required to enable continued profitability and further growth.
In this report, it is proposed that related diversification is the optimal corporate strategy for an established business, ensuring the company continues to profit from technology.

Diversification is a corporate strategy set on the simultaneous departure from the present product line and market structure \cite{ansoff1957strategies}.
Businesses have differing approaches to diversity broadly put into three categories; specialism, related diversification and unrelated diversification. 
%A specialist company is one that is yet to diversify from it's core product and market \cite{johnson2008exploring}. 
Many companies start with a breakthrough product or technological innovation, and a specialised strategy chooses to keep this product as the sole focus of the business \cite{johnson2008exploring}.
At the other end of the spectrum is unrelated diversification.
This strategy involves a company choosing to depart completely from the sector it initially operated in \cite{johnson2008exploring}, and is often referred to as the conglomerate strategy.
In between these two extremes is related diversification, where a company enters a new but related market using the skills base and knowledge they have gained in their current field \cite{johnson2008exploring}. 
Although argued to be contextual and subjective, it is generally the agreement of literature that related diversification produces better company performance \cite{johnson2008exploring, SMJ:SMJ334, SMJ:SMJ82}. 
This claim is examined in this report through examples presented both in the invited talks and through individual research.

\inote{HSL - Put our hypothesis here}
\inote{TJS - The hypothesis is described in the two sentences directly above. Is this clear enough or do we need to spell it out better?}

This report comprises 2 sections. % I checked if it should 'comprised of' this is the right phrasing.
The first discusses the invited talks from Domino Printing, ARM and Imagination technologies with thought to the hypothesis. 
Section two gives further case studies both supporting and opposing the hypothesis in order to gain further insight into the impacts of diversification strategies.


%Many technology based companies start with a key breakthrough technology at the core of the business.
% For example, Google started with it's prolific search engine, Sony started building simple radios and Amazon started with the online market place. \inote{References/Better examples.}




