\sect{Section 1 - Invited Talks} \label{Sect1}
%HSL: I have moved this to the intro
%This section contains a reflection on each of the invited talks in the light of the hypothesis of this report. Full notes from the lectures have been recorded in Appendix 4.

\subsect{Domino Printing Sciences PLC - Carl Reynaud - Director of Hardware Development}
%Overview of each of the talks and an assessment of the information presented in light of the report topic 'How does technology make profit?' 
Domino Printing Sciences PLC designs, manufactures and markets a range of printing equipment for a wide range of applications \cite{DominoAnnual}. 
In 2013 Domino achieved a global turnover of \textsterling335.7m from operations in over 120 countries \cite{DominoFactsheet}. 
Domino was founded in 1978 \cite{DominoFactsheet} as a spin out following a project at technology consultancy Cambridge Consultants.
Graeme Minto led the project developing continuous inkjet (CIJ) technology, and consequently founded Domino Printing Sciences following the departure of the project's client \cite{goffin2010innovation}.
This is an example of a company starting with a core technological innovation as the focus, as asserted by Johnson in \cite{ johnson2008exploring}.

By the mid-90s CIJ was reaching technological limits of exploitation. 
Despite the many advantages of the technology, customers wanted higher resolution images in bigger sizes than were possible with CIJ. Hence Domino began an extensive review of alternative technologies including laser printing, `binary' inkjet and `drop-on-demand' technology.
During the 90s, Domino built on the expertise built from CIJ technology to expand and diversify to become a multi-technology company. 
This diversification into other related technologies have enabled Domino to retain their existing customer base while entering new and emerging markets \cite{goffin2010innovation}.
Instead of just printing date codes, Domino has diversified into technologies to print patterns for laminates, ceramic floor tiles, complex food labels, cable labelling and more.
This is an example of a company adopting related diversification as a corporate strategy, enabling continued and extensive profit from technology.

\subsect{ARM - John Biggs - Consultant Engineer and Co-Founder}
%Overview of each of the talks and an assessment of the information presented in light of the report topic 'How does technology make profit?' 
ARM was founded in 1990 in Cambridge, U.K. as a result of a joint venture between Apple, Acorn and VLSI.
The primary function of ARM was to design processor architectures, but not to manufacture them.
ARM's business model involved the licensing of architectures, as opposed to the sale. 
This also included income from royalties which funded the next products.
They are now the world's leading semiconductor IP company and have shipped 50 billion devices to date.

ARM were able to grasp the low power market ahead of larger CPU firms such as Intel. 
This gave ARM an advantage in the mobile age as it increased battery life of smartphones and tablets.

Although ARM mainly design CPUs, there is still a large amount of diversity within their product range.
In 1997, the company expanded into memory, video and I/O controllers to support the processors.
Latest processors can be found in a range of devices, from $mm^{3}$ sized devices, to $km^{3}$ sized. 
ARM is also currently expanding into the microserver market.
This related diversity, although in very similar areas, resulted in ARM being a resounding success.


\subsect{Imagination Technologies - Dr David Knox - Senior Director Software Engineering}
Overview of each of the talks and an assessment of the information presented in light of the report topic 'How does technology make profit?' 

\subsect{Reflection on Common Themes}
Short conclusion identifying the common themes, or conflicting strategies from the talks.
Summarise in the light of our hypothesis.



