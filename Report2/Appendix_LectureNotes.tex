%  Appendix_LectureNotes.tex
%  Document created by seblovett on seblovett-Ubuntu
%  Date created: Sun 16 Feb 2014 09:34:23 GMT
%  <+Last Edited: Sun 16 Feb 2014 09:35:47 GMT by seblovett on seblovett-Ubuntu +>
% !TeX spellcheck = en_GB
% !TeX root = Report.tex
\phantomsection
\addcontentsline{toc}{section}{Appendix 5 - Notes from Invited Presentations}
\sect{5. Notes from Invited Presentations} \label{Notes}
These notes are raw and not altered in any way from when they were taken from the invited presentation. These notes have been distilled and focussed through the lens of our report title and hypothesis to the content shown in section 1.

\subsect{Lecture 1 - Gary Steele - Fabless Chip Company Start-ups, examples and funding}
Graduated from Southampton in the 80s with a degree in electronics. 
Spent some time in the fabless chip market place which will be a focus of the presentation. 
Worked at National Semiconductor and at ES2. 
ES2 founded in 1985 is the largest European start-up with 100m dollars raised and a wafer fab built in France. 
In 1990 he started the first dedicated Fabless Chip Company called Acapella Ltd based at Chilworth Science Park, eventually sold it to Semtech which is a NASDAQ company for a fair profit. 
1995 started Microcosm Communications Ltd which was truly venture capitalist backed.
 Had two rounds of venture one for 2m dollars and the other for 5.5m dollars. 
Around 16m dollars in todays money which after 5 years was sold to Conexant for 160m dollars which was a very good return (10-20 times return).
Retired etc but then started Phyworks Ltd also in Bristol with seed finance, it was bought out by Maxim shortly. 
Nanotech was started in 2003 and sold to Gennum in 2011 for a 2-3x return on investments.

Until 1990 the wafer fab and design processes were intimately linked. 
The process was unstable and the designs would be tweaked to fit the process. 
Also there was limited CAD support which resulted in poor simulation models and no 3D. 
TSMC in Taiwan was set up in Taiwan and standardised the Philips Semis CMOS processes.
 Large scale mass production leads to 30\% net profit on 20bn dollar revenues. 
There was a high demand for this predictable process flow. Also alongside CAD tools such as those from Cadence. 
These have allowed these to seperate design from Fab. Other foundries include SMIC China, UMC Taiwan, Silterra Malaysia, Global Foundry and IBM. 
Design companies pick the wafer fab that fits their design best and then an assemply and test house.

ARM is one of the most successful companies adopting a design based business model. 
The IP model seems very attractive. However we should be wary of basing these assumptions on one or two data points. 
When comparing the ARM IP model with Die sale price and chip sale price makes more profit, despite being less than Gross Profit Margin Percentage (GPM\%). 
This is since the chip or die is more valuable than the IP and can still be sold in the same volume. Maxim and other good analogue design companies GPM around 70\% accross the business. 
Many more smaller volume orders than very high volume.

Example of product development - Fibre Optic Broadband.
 Originally low volume market with just sub-sea long distance communications considered but as this technology is reaching the home level, the volume rapidly increases. 
The transceiver PCB was one of the important areas of development. 
The development of Transimpedance Amplifiers in CMOS was a novel product. 
How did they protect their advantage? This saved around 2m dollars on around a 10m dollar spend. 
The design was also much simpler designs and the new technology can go through a much quicker fab process due to the more advanced CMOS processes. 
CMOS usually has better power characteristics, allowing some power savings.

When talking about Transimpedance Amplifier how did you protect your capability?

500k dollar design cost and another 500k manufacturing and development cost so around 1m dollar investment to start. 
After 5 years the cumulative gross profit was around 15m dollars.
 How do you get the 1m dollars in the first place. 
It took 2 years for Nanotech semiconductor to break even.

\textbf{Funding Methods}

\textit{Banks} - No

\textit{Organic} from 1 pound to 100k pounds. Works by reinvesting profits for example a design consultancy. This requires slow growth given the 3 year chip cycle and 1m dollar funding required.

\textit{Angel investors} from 100k to 1m pounds. The Government has a form of EIS tax relief for Angel investors. Can halve the true cost of investment and any capital gain is not taxed. However  this still implies a modest growth.

\textit{Venture Capital} from 250k seed to 5-10m pounds. Normally happens in several rounds.

\textit{Venture Debt} similar to VC but a hybrid with debt. This is a way of reducing the stake the venture capitalist own of your business. The interest rate usually 10-13\% which is quite high.

\textit{Customer Co-funding} usually with a lead customer for around 2m dollars. Customer funding of several different IC designs. Perhaps includes some exclusivity agreements.

1 in 300 business plans are funded by venture capital. Europe and the UK chip sector has been a relatively modest success. There are around 10-15 VC funds. Often an annual or 18 month cycle for a VC funded company. VCs need a company to have a big enough market, normally asking for a 1bn dollar market. Typically have a 1 in 10 success rate, which is why the market needs to be big enough to make the whole business be worthwhile. The successful one can often make a yield of 10+ times the initial investment. Around a 7 year commitment required to setup and run a successful VC funded business. Persistence is important. Initial Public Offering or Trade Sale, usually trade in the UK although IPO usually makes higher paper wealth. The paper wealth is difficult to change in to actual cash money though. 

Thoughts on the future. How can EU compete with China and India with huge educated populations. Large problem with the billion dollar market problem where only a very few SoC companies can be successful. This is hard for VCs to make money out of. Probably analogue, mixed signal and RF is likely to make a successful VC funded business.

\textbf{Questions}

How do you identify new markets? The fibre optic market kind of just came out of the blue. The company had a certain expertise after a customer initiated market venture.

Patents - Done both. Pros and cons to each. No patents they had no money so couldn't afford to. If you have enough breakthroughs and go quickly enough, you can beat the incumbent. This first to market undercutting is a single event. You win by being first and running fastest so you don't need patents. In more cash rich companies, you can afford to create it, but can you afford to protect it? If a bigger company has a better law firms? What about defending a patent? 

Is it stressful? Yes.

\subsect{Lecture 2 - David Parker - Southampton Photonics}
Creating bridges between Science and Technology, Technology and products and money. CEO of Southampton Photonics since 2002. 

Are British Airways and Microsoft competitors? Yes or no? They are both competitors in the communication market. They both compete for your leisure dollar and business dollar. Microsoft allows you to do business remotely while BA allow you to do it face to face.

Are Waitrose and Aldi competitors? No they are clearly in the same market, but they have a different target customer. They are more adjacent competitors rather than direct. This is called market segmentation. 

Customers are key. They are the binding force of the business. Expectations are varied, the customer wants the lowest price but the company wants to charge the highest. Returning to the Waitrose/Aldi case study - both deliver their customer segment the best `value' but the expectation of value between the market segments is different. It is easy to get these confused within business. Looking at Apple, the innovation from Apple has changed the market more than any of their competitors.

Some historical context for discussing this includes looking at the dotcom and telecom boom and then the Lehman-crisis. Business is returning to basic business fundamentals. The impacts of the recession (are we still in it?) push forwards business fundamentals and to look at new fundamentals more stringently. Flexibility in the cost base has become more important. How do you create a business model that is flexible so that you can make loads of money in the boom periods, and not go bankrupt in the down-turns. Consolidation is inevitable - for the better or worse - is fuelled out of strategy, accelerated growth or to deliver shareholder value. Investment across the board is generally low.

Mind the Gap. Science/Technology/Product Development/Manufacturing/Channel Delivery/Customer. Between each stage in the product development life-cycle, there are a lot of gaps. For example, at the university level, there are a lot of start-ups. These request investment in their technology, when really it is science as it is not yet a scalable commercial technology. There are many steps to reach a product ready for market. 

Positioning is really important and relates to the Waitrose/Aldi example earlier. This is an example of positioning for a specific customer. Consider vertically integrated companies. They have an opportunity to offer the lowest cost through the absence of margin stacking. However it leads to very large overheads and liabilities. On the flip side, microprocessors have many markets. The question for any emerging business is where you place yourself in the supply chain. Consider the position of core competence. For example the core competence of Aldi is distribution. They have very thin margins, they are very good a getting them from the supplier at the lowest cost and shifting it to the consumer for the lowest cost at the highest volume. Waitrose have a core competence in service, although there is still a requirement in distribution. What is Apple's core competence? Apple has a strong design competence, whether aesthetic or in the internals or in the user experience. They also have a core competence in marketing, both directly within shops etc and indirectly through building a strong brand. Going back to BA and Microsoft. Microsoft obviously have a core competence in software and BA have a core competence in logistics. Southwest airlines had a core competence in logistics. They were really the first low cost airline. They asked their customers what was the most important thing about airlines, and the response was safety and arriving on-time. Hence they did things like taking away boarding passes. There was no money making issue here it was to stop people turning up late. This gave them an edge allowing them to leave on time consistently. The logistics are improved. Pick core competencies wisely and make the rest consistent.

High volume optical components for fibre optic communications. At first it was very vertically integrated as it was so cutting edge. It took in raw materials and sold to network equipment manufacturers like Cisco. After a while there was increased competition. Reducing costs can only get so low by increasing volume, step changes in cost can be made through inventive steps. The core competence had no clear focus when it was vertically integrated. As the market grew, the company shifted its competence and made sure it was consistent in it's behaviour. 

SPI Lasers - or Southampton Photonics Incorporated - was spun out of Southampton University to develop the Bragg rating which is a type of optical filter. In 2002 the telecom industry collapsed. Eventually decided to refocus the business strategy. The core competence was in glass fibre with a strong IP portfolio and clever people. However, it had this competence but no market. At the time there was a new emerging market for fiber lasers. The company could apply its core competence in this slightly different market.

Perpetuum example. They developed small energy harvesting generators, for example to run a sensor on a pump or something. However it never sold any products. This was because it was so far down the supply chain, the tech was good but it was too far removed from the customer. Actually selling into the rail industry now which has a real problem with monitoring, ie it had a customer. The core competence was in energy harvesting, but a secondary competence was required to interpret the data. Very heavily protected - adds a barrier to entry as there are no wires. The company used to be a component supplier, but then became a system platform, does the company turn into a full service provider? 

Patenting hardware is often very effective, software less slow. Perpetuum made 200k in 2012, then 2mill in 2013, then more maybe this year?

Feanium? Another Uni spin-out. Handful of lasers sold a week, very specialist market segment. Danger of it becoming high volume and how that business model can cope with that?


\subsect{Lecture 3}

\lipsum[3] 

