%  Appendix_LectureNotes.tex
%  Document created by seblovett on seblovett-Ubuntu
%  Date created: Sun 16 Feb 2014 09:34:23 GMT
%  <+Last Edited: Sun 16 Feb 2014 09:35:47 GMT by seblovett on seblovett-Ubuntu +>
% !TeX spellcheck = en_GB
% !TeX root = Report.tex
\phantomsection
\addcontentsline{toc}{section}{Appendix 5 - Notes from Invited Presentations}
\sect{5. Notes from Invited Presentations} \label{Notes}
These notes are raw and not altered in any way from when they were taken from the invited presentation. These notes have been distilled and focussed through the lens of our report title and hypothesis to the content shown in section 1.

\subsect{Lecture 1 - Gary Steele - Fabless Chip Company Start-ups, examples and funding}
Graduated from Southampton in the 80s with a degree in electronics. 
Spent some time in the fabless chip market place which will be a focus of the presentation. 
Worked at National Semiconductor and at ES2. 
ES2 founded in 1985 is the largest European start-up with 100m dollars raised and a wafer fab built in France. 
In 1990 he started the first dedicated Fabless Chip Company called Acapella Ltd based at Chilworth Science Park, eventually sold it to Semtech which is a NASDAQ company for a fair profit. 
1995 started Microcosm Communications Ltd which was truly venture capitalist backed.
 Had two rounds of venture one for 2m dollars and the other for 5.5m dollars. 
Around 16m dollars in todays money which after 5 years was sold to Conexant for 160m dollars which was a very good return (10-20 times return).
Retired etc but then started Phyworks Ltd also in Bristol with seed finance, it was bought out by Maxim shortly. 
Nanotech was started in 2003 and sold to Gennum in 2011 for a 2-3x return on investments.

Until 1990 the wafer fab and design processes were intimately linked. 
The process was unstable and the designs would be tweaked to fit the process. 
Also there was limited CAD support which resulted in poor simulation models and no 3D. 
TSMC in Taiwan was set up in Taiwan and standardised the Philips Semis CMOS processes.
 Large scale mass production leads to 30\% net profit on 20bn dollar revenues. 
There was a high demand for this predictable process flow. Also alongside CAD tools such as those from Cadence. 
These have allowed these to seperate design from Fab. Other foundries include SMIC China, UMC Taiwan, Silterra Malaysia, Global Foundry and IBM. 
Design companies pick the wafer fab that fits their design best and then an assemply and test house.

ARM is one of the most successful companies adopting a design based business model. 
The IP model seems very attractive. However we should be wary of basing these assumptions on one or two data points. 
When comparing the ARM IP model with Die sale price and chip sale price makes more profit, despite being less than Gross Profit Margin Percentage (GPM\%). 
This is since the chip or die is more valuable than the IP and can still be sold in the same volume. Maxim and other good analogue design companies GPM around 70\% accross the business. 
Many more smaller volume orders than very high volume.

Example of product development - Fibre Optic Broadband.
 Originally low volume market with just sub-sea long distance communications considered but as this technology is reaching the home level, the volume rapidly increases. 
The transceiver PCB was one of the important areas of development. 
The development of Transimpedance Amplifiers in CMOS was a novel product. 
How did they protect their advantage? This saved around 2m dollars on around a 10m dollar spend. 
The design was also much simpler designs and the new technology can go through a much quicker fab process due to the more advanced CMOS processes. 
CMOS usually has better power characteristics, allowing some power savings.

When talking about Transimpedance Amplifier how did you protect your capability?

500k dollar design cost and another 500k manufacturing and development cost so around 1m dollar investment to start. 
After 5 years the cumulative gross profit was around 15m dollars.
 How do you get the 1m dollars in the first place. 
It took 2 years for Nanotech semiconductor to break even.

\textbf{Funding Methods}

\textit{Banks} - No

\textit{Organic} from 1 pound to 100k pounds. Works by reinvesting profits for example a design consultancy. This requires slow growth given the 3 year chip cycle and 1m dollar funding required.

\textit{Angel investors} from 100k to 1m pounds. The Government has a form of EIS tax relief for Angel investors. Can halve the true cost of investment and any capital gain is not taxed. However  this still implies a modest growth.

\textit{Venture Capital} from 250k seed to 5-10m pounds. Normally happens in several rounds.

\textit{Venture Debt} similar to VC but a hybrid with debt. This is a way of reducing the stake the venture capitalist own of your business. The interest rate usually 10-13\% which is quite high.

\textit{Customer Co-funding} usually with a lead customer for around 2m dollars. Customer funding of several different IC designs. Perhaps includes some exclusivity agreements.

1 in 300 business plans are funded by venture capital. Europe and the UK chip sector has been a relatively modest success. There are around 10-15 VC funds. Often an annual or 18 month cycle for a VC funded company. VCs need a company to have a big enough market, normally asking for a 1bn dollar market. Typically have a 1 in 10 success rate, which is why the market needs to be big enough to make the whole business be worthwhile. The successful one can often make a yield of 10+ times the initial investment. Around a 7 year commitment required to setup and run a successful VC funded business. Persistence is important. Initial Public Offering or Trade Sale, usually trade in the UK although IPO usually makes higher paper wealth. The paper wealth is difficult to change in to actual cash money though. 

Thoughts on the future. How can EU compete with China and India with huge educated populations. Large problem with the billion dollar market problem where only a very few SoC companies can be successful. This is hard for VCs to make money out of. Probably analogue, mixed signal and RF is likely to make a successful VC funded business.

\textbf{Questions}

How do you identify new markets? The fibre optic market kind ofjust came out of the blue. The company had a certain expertise after a customer initiated market venture.

Patents - Done both. Pros and cons to each. No patents they had no money so couldnt afford to. If you have enough breakthroguhs and go quickly enough, you can beat the incumbant. This first to market undercutting is a single event. You win by being first and running fastes so you dont need patents. In more cash rich companies, you can afford to create it, but can you afford to protect it? If a bigger company has a better law firms? What about defending a patent? 

Is it stressful? Yes.

\subsect{Lecture 2}

\lipsum[2]

%%%%%%%%%%%%%%%%%%%%%%%%%%%%%%%%%%%%%%%%%%%%%%%%%%%%%%%%%%%%%
%% IMAGIGAYTION LECTURE NOTES
%%%%%%%%%%%%%%%%%%%%%%%%%%%%%%%%%%%%%%%%%%%%%%%%%%%%%%%%%%%%%
\subsect{Lecture 3}

\lipsum[3] 

