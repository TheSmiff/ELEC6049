% !TeX spellcheck = en_GB
% !TeX root = Report.tex
\phantomsection
\addcontentsline{toc}{subsection}{Iridium}
\subsect{Iridium}

In 1998 Iridium, a company that spun out of Motorola in 1991, launched their telecommunications service to the public.
In 1999 Iridium filed for bankruptcy.
The company started a ground breaking Low Earth Orbiting (LEO) satellite network which was made possible by a total \$$5$ billion investment from Motorola.
Expanding on existing mobile phone technology by offering global connections without delays.
Satellite technology is extremely complex and the seven years before release was spent designing, constructing and launching the network~\cite{fink2000iridium}. 
It is widely agreed that the failure of the company can be attributed to the expensive handsets and call tariffs.
Initially they managed to attract less than half the number of expected customers and these fell quickly as issues with the handsets arose~\cite{bill1999}.
Iridium has made a comeback since this fatal chain of events and currently working to upgrade the network under the new brand Iridium NEXT~\cite{iridiumNEXT}.
Resolving their previous issues by aiming services at industries, such as maritime and oil, that can make far better use of a truly global network~\cite{better2009}.

Iridium holds a large amount of patents which were spawned around the original idea, US Patents 5,410,728 \& 5,604,920, conceived by three engineers~\cite{ip2010,bertiger1995satellite,bertiger1997satellite}.
This aggressive patenting strategy evidently did not protect the company from failure and the argument can be made that patents were redundant in their approach.
The concept was, and in fact still is, on the very crest of the technology wave and required a very large amount of backing to implement.
No one else 
