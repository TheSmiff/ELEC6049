
%  Document created by seblovett on seblovett-Ubuntu
%  Date created: Sun 16 Feb 2014 13:54:36 GMT
%  <+Last Edited: Sat 22 Mar 2014 12:38:54 GMT by seblovett on seblovett-Ubuntu +>
% !TeX spellcheck = en_GB
% !TeX root = Report.tex
\phantomsection
\addcontentsline{toc}{subsection}{Atmel}
\subsect{Atmel}


%\todo{Brief intro to atmel.}
%\todo{Show that the classify as a blossoming company during the time we are looking at.}
Atmel Technologies was founded in 1984 by a former Intel employee, George Perlegos.
They were venture capital funded, with only \$30,000 in initial investment.
Atmel in the current day are a very well established company, with offices in many different countries, multiple product ranges and had a turn over of \$150.93 million in the last quarter of 2013 \cite{atmel:profit}.
For the purpose of this case study, the early life of Atmel provides a good case study with respect to their patents.


Atmel hold around 1200 patents as of 2014 \cite{atmel:patents} in varying aspects of technology, from touch screen related methods, to fabrication techniques.
Their first patent was issued on 16th May, 1989 \cite{atmel:eprompatent}, 5 years after they started out.
In the years previous to their first patent, Atmel encountered patent issues.

In 1987, Atmel were hit with a patent infringement case from Intel.
Intel claimed that Atmel's ERPOM technology which they were selling to Motorola and Nokia, infringed on the patents that they held.
Atmel could have fought a legal battle against Intel, but they chose not to. 
This could have been due to the costs (given that Atmel did not start out with a lot of money, legal costs were likely to have made a detrimental impact on the company), or that Atmel did not have much of a defence.
As Atmel didn't own any patents at this time either, it looked like Intel would win the legal case. 

With hindsight, the decision to not fight the legal battle was the correct one. 
Little money was spent on the proceedings \todo{find out what Atmel had to do to compensate Intel}, and it forced Atmel to redesign their EPROM technology.
Their new design turned out to not only out perform Intel's device, it also consumed less power.
Later, this improved memory was included in their microcontrollers, providing an edge in a different market as the EPROM could be included on chip.
Atmel's first patent was about their EPROM fabrication process \cite{atmel:eprompatent}.
This patent was then followed by 3 more that year, all based around their EPROM technology \cite{atmel:eprom1,atmel:eprom2,atmel:eprom3}.

This case study shows that the acquisition of patents is not necessarily a good thing.
If Atmel had gained early patents of their EPROM technology, they may have been more likely to fight a costly legal battle against Intel.
Given the little funding Atmel started with, it could have cost them more than money, with time being wasted. 
It could be argued that the initial technology did in fact, infringe on Intel's patents, and could therefore not be patented.
However, due to the financial standings of the company, the costs for both owning and defending a patent were possibly too much at the time.
Even if Intel did not hold patents, and did not prosecute Atmel, the overall success of Atmel could be debatable. 
Their renewed effort on memory devices impacted their later microcontroller memory, of which is now a very large core competence of Atmel.

Since their success with EPROM memory, Atmel received extra funding \cite{atmel:acq1} and purchased foundries to further their R\&D into silicon memories. 
In the modern market, Atmel develop both hardware and software products. 
Their initial success was spurred by not having a defence against someone else's patents. 
In this situation, patents would have not been advantageous to Atmel.

