% !TeX spellcheck = en_GB
% !TeX root = Report.tex
\phantomsection
\addcontentsline{toc}{subsection}{Gary Steele \& Nanotech Semiconductor Ltd.}
\subsect{Lecture 1 - Gary Steele \& Nanotech Semiconductor Ltd.}
Nanotech Semiconductor Ltd provides a case study of a successful emerging technology company. 
The company was founded in 2003 by prolific entrepreneur Gary Steele, following great success with his Acapella Ltd and Microcosm Communications Ltd business ventures \cite{GazNesta}.
Nanotech was a fabless IC company with expertise in CMOS analog and mixed-signal designs for the fibre communications industry, and is cited in \cite{GazLightwave} as being the fastest growing start-up in the fibre-optics IP market in 2010.
The company was sold to Gennum Corporation in 2011 for \$34 million (USD) \cite{GazBloom}, which was subsequently acquired by Semtech in 2012.
 
One of the company's key innovations was transmitter/receiver chips for fast Ethernet links.
The uniquity of the design was achieved by combining the transimpedance amplifier and the limiting amplifier into a single chip.
This required significant progress in noise suppression.
In \cite{GazEW}, Steele is quoted, ``We put a lot of intellectual property into the noise suppression.''
This gives an insight into the role of patenting within Nanotech.
Patenting new technology is preferable, since it protects the innovation and maintains a competitive advantage.
%\todo[color=red]{This has been mentioned in the intro. Remove?}
However, as an emerging company, capital to invest in patenting was not always available.
This leads to the selective approach to patenting, whereby only strategic patents for the most commercially critical technology was obtained.
The patent strategy for Nanotech was a compromise between adequately protecting the technological innovations and capital cost.

The role of patents in the exit strategy of an emerging company is also an important aspect for consideration.
While Nanotech did not have a vast patent portfolio, it can be seen through patent history the migration of patents from Nanotech Semiconductor to Semtech Corporation \cite{nuttgens2013closed1} \cite{nuttgens2013closed2}.
Emerging companies can be acquired by larger established companies for a number of reasons, one of which is the ownership of a high-profile IP portfolio.
The Acquisition of Nanotech Semiconductor by Gennum Corporation can be partly attributed to synergistic expertise and technology portfolios \cite{GazBloom}, however the role of IP in acquisitions should not be overlooked.
 
