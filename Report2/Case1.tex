% !TeX spellcheck = en_GB
% !TeX root = Report.tex
\phantomsection
\addcontentsline{toc}{subsection}{Case1}
\subsect{Nest Labs - Patents for protection and adding value}

Nest Labs was founded in May 2010 by Tony Fadell and Matt Rogers \cite{NestFactsheet}.
The company designs innovative smoke alarms and thermostats that feature iterative learning and are internet connected.
From humble beginnings, Nest have now sold 14m thermostats and 50m smoke alarms and employ over 300 people \cite{NestFactsheet}.
The company was acquired by Google for \$3.2bn cash \cite{NestReuters} in February 2014 \cite{NestFactsheet}.
Although the company has not yet existed for 8 years, the acquisition by Google makes Nest Labs a successful emerging company.
This case-study will examine the acquisition by Google and ascertain the criticality of patents to the success of this emerging company.

Nest has a strong design heritage, both co-founders were ex-Apple employees and CEO Tony Fadell is widely accredited as the father of the Apple iPod \cite{NestAppleInsider}.
Many Nest employees are also Apple alums \cite{NestReuters}.
This includes Richard `Chip' Lutton, appointed as Vice President and general counsel in 2012 \cite{NestAppleInsider}, who was heavily involved in Apple's patent strategy \cite{NestReuters}.
Nest has a somewhat unusual patent strategy for an emerging company.
Before the Google acquisition Nest had already settled 100 patent disputes \cite{NestCoLabs}.
It also owns 200 patents and has another 200 ready to file \cite{NestCoLabs, NestIV}.
The company signed an agreement with Intellectual Ventures in September 2013 granting access to IV's 40,000 strong patent portfolio \cite{NestIV}.
This is an aggressive approach to patenting, which is not normally pursued by an independent emerging company due to the legal costs \cite{zahra1996technology}.
The expertise gained through some of the employees experience at Apple must play a role in Nest's patenting strategy.

Nest's aggressive patent strategy is partly motivated by defence. 
The company has faced three major patent disputes since it's inception.
Thermostat manufacturer Honeywell filled a lawsuit against Nest for patent infringement in 2012 \cite{NestHoneywellGig, NestHoneywellVerge, NestHoneywell}.
The claims surrounded several patents for technology in Nest's intelligent thermostat.
Although the lawsuit is ongoing \cite{NestFirstVerge}, Nest has largely dismissed these claims, and have stated that Honeywell often use this strategy whenever posed with competition.
As a result of this first lawsuit, Nest hired Richard Lutton and entered the deal with Intellectual Ventures \cite{NestFirstGig}.
Additional patent infringement cases have been filed by First Alert in November 2013 over Nest's intelligent smoke alarm \cite{NestFirstVerge, NestFirstGig}.
Allure Energy are also pursuing a lawsuit for proximity control technology in Nest's thermostat \cite{NestAllureCnet}.
Taking an aggressive approach to patenting will allow Nest to defend itself against these more established companies \cite{NestAllureCnet}.
Disruptive technology such as the Nest thermostat has the ability to transform a marketplace, and stall major revenue streams for incumbants \cite{NestCoLabs}. 
Larger established companies often use the threat of expensive patent litigation to manage the threat of emerging disruptive technology, as Honeywell, Allure and First Alert have in this case study.
While a defence strategy can be expensive, when combined with an appropriate level of expertise and financial backing, an emerging company can adequately defend itself in court, allowing the company to profit from it's technology.






%Nest patent litigation with Honeywell and Allure

%Nest patent acquisition by Google

