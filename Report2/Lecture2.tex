% !TeX spellcheck = en_GB
% !TeX root = Report.tex
\phantomsection
\addcontentsline{toc}{subsection}{David Parker - Southampton Photonics}
\subsect{Lecture 2 -  David Parker \& Perpetuum}


%\inote{Give a brief intro to them}
Perpetuum are a Southampton based business specialising in energy harvesting modules. 
The company was founded in 2004 and began life as a component supplier for their power module.
\todo[color=red]{Needed?}Perpetuum do not classify as an emerging company in the current day, but the early life of the company and their patent acquisitions are discussed here. 
They are a venture capital funded start-up involving multiple companies.
It was a spin off from a technology developed at Southampton University. 
%\inote{Show Perpetuum is a start up.}

%\inote{Discuss patent acquisitions of Perpetuum}
%http://patents.justia.com/assignee/perpetuum-ltd?page=2
Perpetuum currently hold 22 patents to date. 
Their first was issued on the 23rd August, 2007, some 3 years after they began.
These initial patents prevented any other businesses from being able to reproduce their technology. 
However, the company initially struggled with business.
Their target market was the Oil and Gas industry as they use large drills which vibrate, making it a perfect spot to house their module.
The main issue was that the target market had no application for the module that Perpetuum were attempting to supply. %\todo{find out what exactly they were initially supplying}

%In 20xx \todo{find when Parker joined}, David Parker was asked to join Perpetuum. 
Due to the business not fulfilling their potential, they undertook a change. 
The core competencies of the company were considered.
It was easy to see that the main core competency of Perpetuum was in developing energy harvesting modules. 
A second core competence was added to the company of developing a wireless sensor for their module, transferring the product developed from a component, to a system.
This change in focus was accompanied by a change in market - the system was aimed at the transport industry, particularly trains, where there was a gap for a remote monitoring system of the bearings and suspension. 

Even though Perpetuum had a great, novel idea with patents to protect them, it did not make them money by owning the patents alone.
The patents definitely helped, as it meant no other company could have undermined them during the time they were pursuing the non-existent market in oil and gas.
The patents here acted as a barrier to the market, giving them the time to find the correct opportunity to become a success.
Perpetuum have grown rapidly since the shift in core competence and market, with a turn over of around \pounds 2 million in 2013.
There is no doubt that Perpetuum will become ever more successful through the combination of a good idea in the correct market, with patents to protect their product. 

%Another interesting point raised in the lecture by Dr Parker, was that IP is only effective in hardware industry.
%Software is difficult to patent \todo{why is software difficult to patent? Refer to the introduction on patents}

%\inote{Discuss that even though they had patents, they were in the wrong market.}

%\inote{Now they are in a good market, their patents are protecting them and they are doing well.
%Define ``well''. 
%Are they established yet?}

%\inote{Parker briefly spoke about that software is difficult to patent, but hardware is very effective. 
%Results in us looking at hardware companies rather than software.}
