%  Appendix_Minutes.tex
%  Document created by seblovett on seblovett-Ubuntu
%  Date created: Sun 16 Feb 2014 09:32:52 GMT
%  <+Last Edited: Sun 16 Feb 2014 09:33:34 GMT by seblovett on seblovett-Ubuntu +>
\phantomsection
\addcontentsline{toc}{section}{Appendix 2 - Minutes from Kick-off Meeting}
\sect{2. Meeting Minutes - Kick-off Meeting}
\begin{center}
\begin{longtable}{| m{0.2\textwidth} | m{0.6\textwidth} |} \hline
\textbf{Purpose} & ELEC6049 Team Kick-Off Meeting \\ \hline
\textbf{Date and Time} & Tuesday 4th March 2014 at 12:40 \\ \hline
\textbf{Venue} & 1st floor Murray Building, Highfield Campus \\ \hline
\textbf{Participants} & TJS (Tom Smith), HSL (Henry Lovett), AJR (Ashley Robinson)\\ \hline
\textbf{Apologies} &None \\ \hline
\multirow{4}{*}{\textbf{Agenda}} & Assign Chair for this report. \\
 & Generate initial ideas for research. \\ 
 & Agree expectations of work and schedule. \\
 & Agree date and agenda of next meeting. \\ \hline
\end{longtable}
\end{center}

\subsect{Minutes of the Meeting}
\begin{center}
\begin{longtable}{| p{0.05\textwidth} |>{\raggedright\arraybackslash}p{0.15\textwidth} | p{0.5\textwidth} |>{\raggedright\arraybackslash}p{0.175\textwidth}|} \hline
\textbf{ID} & \textbf{Subject} & \textbf{Notes and Discussion} & \textbf{Action} \\ \hline
\endhead
1.0	&	Chair	&	The group decided that HSL should be the chair for this report. AJR is still pursuing job applications and will hence chair the final report. 	&   -	 \\ \hline
2.0	&	Feedback & The feedback from the first report was good. The group have decided to look more at strategy rather than the history of a company. & - \\ \hline
3.0 & Start Up Company & A definition from literature is needed as to how to classify a start up company. All are to look for ideas. &  \textbf{ALL A1.0} \\ \hline

4.0 & Initial Ideas & Good ideas make money in a start up. However, how does one define a ``good'' idea? What drives desirability for products. & - \\ \hline
5.0 & Kickstarter & Kickstarter was deemed unproductive - it is not how a start up makes money, mearly a mechanism for excelling good ideas & - \\ \hline
6.0 & Hypothesis 1 & Need a good idea and a good market. HSL raised point of RAMBUS which was both a good idea and had a good market, but was unsuccessful due to poor management decisions. Also unknown if this was a start up at the time, and if this even applies to the hypothesis suggested. & - \\ \hline
7.0 & Hypothesis 2 & Patents, publicity or trade secrets. Start ups that manage to get patents for a desirable technology, whether it is applicable at the time or future, could become desirable to larger companies to fund. HSL raised example of Quantum Research Group who were bought out by Atmel. & - \\ \hline

8.0 & Deadline & TJS raised that the deadline is 3 weeks today - 25th March. & - \\ \hline

9.0 & Research & Ideas have been raised, but TJS suggested we research and have a clear starting point and idea before looking at the case studies. Due to Zepler outage, disabling most of us doing most other work, more effort will be put into this module this week to compensate & \textbf{ALL 2.0} \\ \hline

10.0 & Git & Git has proved useful. A new report template is to be set up. & \textbf{HSL 3.0} \\ \hline

11.0 & Next meeting & Group decided to meet \textbf{5th March} as no one can do other work so will concentrate on this. & - \\ \hline

\end{longtable}
\end{center}

\subsect{Action List}
\begin{center}
\begin{longtable}{| p{0.05\textwidth} | >{\raggedright\arraybackslash}p{0.15\textwidth} |  p{0.5\textwidth} | >{\raggedright\arraybackslash}p{0.175\textwidth}|} \hline
\textbf{ID} & \textbf{Action} & \textbf{Comments} & \textbf{Status} \\ \hline
\endhead
A1.0	&	Definition of a `Start Up'	&	Find a definition from literature for what we can classify as a start up company.	& Open 4th March \\ \hline
A2.0	&	Initial Research	&	All to do some research into a potential hypothesis. To be discussed tomorrow.	&	Open 4th March \\ \hline
A3.0	&	Set Git up	&	HSL to set add second report to the Git.	&	Open 4th March	\\ \hline	
\end{longtable}
\end{center}

\subsect{Next Meeting: 5th March 2014, 12:00, Level 3 Zepler, Highfield Campus. If Zepler is inaccessible, meet outside the building and find a room elsewhere on campus. }

%\clearpage

\phantomsection
\addcontentsline{toc}{section}{Appendix 3 - Minutes from Progress Meeting}
\sect{3. Meeting Minutes - Progress Meeting}
\begin{center}
\begin{longtable}{| m{0.2\textwidth} | m{0.6\textwidth} |} \hline
\textbf{Purpose} & ELEC6049 Report Meeting \\ \hline
\textbf{Date and Time} & Wednesday 12th February at 11:00 \\ \hline
\textbf{Venue} & 3rd floor Zepler Building, Highfield Campus \\ \hline
\textbf{Participants} & TJS (Tom Smith), HSL (Henry Lovett), AJR (Ashley Robinson)\\ \hline
\textbf{Apologies} & None \\ \hline
\multirow{5}{*}{\textbf{Agenda}} & Update on research status. \\
 & Discuss case study examples. \\ 
 & Read any report stubs as a group. \\
 & Identify and allocate work to finish report. \\ 
 & Agree next meeting. \\ \hline
\end{longtable}
\end{center}

\subsect{Minutes of the Meeting}
\begin{center}
\begin{longtable}{| p{0.05\textwidth} |>{\raggedright\arraybackslash}p{0.15\textwidth} | p{0.5\textwidth} |>{\raggedright\arraybackslash}p{0.175\textwidth}|} \hline
\textbf{ID} & \textbf{Subject} & \textbf{Notes and Discussion} & \textbf{Action} \\ \hline
\endhead
1.0	&	House Keeping	&	HSL identified and addendum to the minutes of the last meeting and has corrected the typo. 
						Subsequent to this the minutes of the last meeting were adopted by the group								&	- \\ \hline
2.0	&	Tutorial Session	&	TJS attended the non-official tutorial session with Chris Freeman (CF). 
						TJS fed back to the group that CF had been very positive regarding the hypothesis proposed. 
						CF said it looked like a good report with good examples to back it up, the ideas were received with considerable enthusiasm. 
						CF highlighted the importance of good and appropriate referencing which should be heeded.						&	- \\ \hline
3.0	&	Research Progress	&	HSL provided some initial reading on Google, Amazon and Apple with particular emphasis on ZDnet article.
						AJR questioned if this is a credible source. 
						The group agreed it is important to follow reference trails through literature to establish the original source.			 &	- \\ \hline
4.0	&	Report Progress	&	TJS has drafted an introduction to the report and written the Domino Printing section. 
						The group read the report so far. 														 & 	   \\
4.1     &        				&	HSL noted that IEEE allows you to refer to numbered references as nouns, which the group agreed should be followed despite the report not being explicitly IEEE style.  &	\textbf{TJS A4.0} \\ 
4.2	&				&	HSL suggested moving the introductions from the start of each section to the introduction of the report - Actioned.		&\\
4.3	&				&	TJS proposed that the Domino Printing case study should be used as a template - Agreed by the group.				&	   \\
4.4	&				&	The group decided on three or potentially four case studies for section 2. 
						Google was chosen as an example of related diversification. 
						It was mentioned in the previous meeting as an example. 
						Atmel was also suggested, and HSL has some insider experience in this area hence justifying the choice. 
						Intel were discussed by the group. 
						Their failure to diversify into mobile must have caused some damage. 
						This will make a good study of a company that has remained relatively specialised. 
						Finally another company on the other end of the spectrum will be chosen. 
						Virgin Group was discussed with operations in finance, travel, space exploration and just about everything. 
						Mitsibushi were also discussed, with unrelated diversification spanning power equipment, cars and nuclear power stations. 
						One of these conglomerate organisations will also form a comparison for section 2. 
						The group agreed that the case studies should be written by the next tutorial, so that then the report could be collated into a coherent document.			&  \textbf{TJS A5.0 \& A8.0, AJR A6.0, HSL A7.0} \\
4.5	&				&	HSL mentioned that the group should break up the \TeX files. 										&  \textbf{HSL A9.0}	 \\ \hline
5.0	&	Remaining Talks	& 	The team agreed that HSL and AJR will take responsibility for one of the remaining invited talks each. 
						This includes keeping lecture notes in the appendix of the report and writing the relevant subsection in section 1.			& \textbf{AJR A10.0, HSL A11.0} \\ \hline
6.0	&	Minutes		&	The group agreed that TJS would collate these minutes.											& \textbf{TJS A12.0} \\ \hline
7.0	&	AOB			&	The next meeting will take place at the tutorial. The meeting was closed. & - \\ \hline

\end{longtable}
\end{center}

\subsect{Action List}
\begin{center}
\begin{longtable}{| p{0.05\textwidth} | >{\raggedright\arraybackslash}p{0.15\textwidth} |  p{0.5\textwidth} | >{\raggedright\arraybackslash}p{0.175\textwidth}|} \hline
\textbf{ID} & \textbf{Action} & \textbf{Comments} & \textbf{Status} \\ \hline
\endhead
A1.0	&	Research	&	All to start research. Use Git Issue to highlight useful research to the group. Make notes of all sources.	& Open 4th Feb - Closed 11th Feb \\ \hline
A2.0	&	Introduction	&	Define an established company. Introduce hypothesis.	&	Open 4th Feb - Closed 11th Feb \\ \hline
A3.0	&	Case Study	&	Each to identify a case study for proposal to the group by next week.	&	Open 4th Feb - Closed 11th Feb	\\ \hline
A4.0   &	IEEE Styling	&	Change reference from author and citation to citation as a noun		&	Open 11th Feb	\\ \hline
A5.0	&	Google Case Study	&	Research Google as a successful implementation of related diversification.	&	Open 11th Feb	\\ \hline 
A6.0	&	Intel Case Study	&	Research Google as a specialised company and the effects of not diversifying.   &	Open 11th Feb	\\ \hline
A7.0	&	Atmel Case Study	&	Research Atmel as a successful implementation of related diversification.		&	Open 11th Feb	\\ \hline
A8.0	& 	Conglomerate Case Study	&	Research a conglomerate (Virgin, Mitsibushi etc) for an example of unrelated diversification. 	&	Open 11th Feb	\\ \hline
A9.0	&	\TeX files	&	Break up the \TeX files into sections so the \LaTeX word count will work properly. 	&	Open 11th Feb	\\ \hline
A10.0	&	ARM Talk	&	Write up the invited talk notes and corresponding Section 1 part.	&	Open 11th Feb	\\ \hline
A11.0	&	Imagination Talk 	&	Write up the invited talk notes and corresponding Section 1 part.	&	Open 11th Feb	\\ \hline
A12.0	&	Minutes	&	Write up the minutes and circulate. & Open 11th Feb \\ \hline
\end{longtable}
\end{center}

\subsect{Next Meeting: 18th Feb 14, in the scheduled tutorial session 12:00 05/2015.}

